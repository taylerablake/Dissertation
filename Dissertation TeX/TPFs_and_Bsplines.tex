\documentclass[12pt]{article}
\usepackage{graphicx,psfrag,amsfonts,float,mathbbol,xcolor,cleveref}
\usepackage{arydshln}
\usepackage{amsmath}
\usepackage{tikz}
\usepackage[mathscr]{euscript}
\usepackage{subcaption}
\usepackage{mathtools}
\usepackage{IEEEtrantools}
\usepackage{amsthm}
\usepackage[letterpaper, left=1in, top=1in, right=1in, bottom=1in,nohead,includefoot, verbose, ignoremp]{geometry}
\newcommand{\comment}[1]{\text{\phantom{(#1)}} \tag{#1}}
\newcommand\numberthis{\addtocounter{equation}{1}\tag{\theequation}}
\newcommand*\needsparaphrased{\color{red}}
\newcommand*\needsproof{\color{blue}}
\newcommand*\outlineskeleton{\color{green}}
\newcommand{\PP}{\mathcal{P}}
\newcommand{\bfeps}{\mbox{\boldmath $\epsilon$}}
\newcommand{\bfgamma}{\mbox{\boldmath $\gamma$}}
\newcommand{\bflam}{\mbox{\boldmath $\lambda$}}
\newcommand{\bfphi}{\mbox{\boldmath $\phi$}}
\newcommand{\bfsigma}{\mbox{\boldmath $\sigma$}}
\newcommand{\bfbeta}{\mbox{\boldmath $\beta$}}
\newcommand{\bfalpha}{\mbox{\boldmath $\alpha$}}
\newcommand{\bfe}{\mbox{\boldmath $e$}}
\newcommand{\bff}{\mbox{\boldmath $f$}}
\newcommand{\bfone}{\mbox{\boldmath $1$}}
\newcommand{\bft}{\mbox{\boldmath $t$}}
\newcommand{\bfo}{\mbox{\boldmath $0$}}
\newcommand{\bfO}{\mbox{\boldmath $O$}}
\newcommand{\bfx}{\mbox{\boldmath $x$}}
\newcommand{\bfX}{\mbox{\boldmath $X$}}
\newcommand{\bfz}{\mbox{\boldmath $z$}}


\newcommand{\bfm}{\mbox{\boldmath $m}}
\newcommand{\bfy}{\mbox{\boldmath $y$}}
\newcommand{\bfd}{\mbox{\boldmath $d$}}
\newcommand{\bfc}{\mbox{\boldmath $c$}}
\newcommand{\bfa}{\mbox{\boldmath $a$}}
\newcommand{\bfb}{\mbox{\boldmath $b$}}
\newcommand{\bfY}{\mbox{\boldmath $Y$}}
\newcommand{\bfS}{\mbox{\boldmath $S$}}
\newcommand{\bfZ}{\mbox{\boldmath $Z$}}
\newcommand{\cardT}{\vert \mathcal{T} \vert}
%\newenvironment{theorem}[1][Theorem]{\begin{trivlist}
%\item[\hskip \labelsep {\bfseries #1}]}{\end{trivlist}}
%\newenvironment{corollary}[1][Corollary]{\begin{trivlist}
%\item[\hskip \labelsep {\bfseries #1}]}{\end{trivlist}}
%\newenvironment{proposition}[1][Proposition]{\begin{trivlist}
%\item[\hskip \labelsep {\bfseries #1}]}{\end{trivlist}}
%\newenvironment{definition}[1][Definition]{\begin{trivlist}
%\item[\hskip \labelsep {\bfseries #1}]}{\end{trivlist}}

\newtheorem{theorem}{Theorem}[section]
\newtheorem{lemma}[theorem]{Lemma}
\newtheorem{proposition}[theorem]{Proposition}
\newtheorem{corollary}[theorem]{Corollary}

\theoremstyle{definition}
\newtheorem{definition}{Definition}[section]
\newtheorem{example}{Example}[section]
\def\bL{\mathbf{L}}


\makeatletter
\renewcommand{\theenumi}{\Roman{enumi}}
\renewcommand{\labelenumi}{\theenumi.}
\renewcommand{\theenumii}{\Alph{enumii}}
\renewcommand{\labelenumii}{\theenumii.}
\renewcommand{\p@enumii}{\theenumi.}
\makeatother

 \begin{document}

\nocite{*}


\section{A representation for piecewise polynomial functions}

Let $\xi = \left\{ \xi_1<\xi_2<\dots<\xi_{l+1} \right\}$ be a set of strictly increasing series of points, and let $k$ be a positive integer. Further, let $P_1,\dots,P_l$ denote a sequence of $l$ polynomials of order $k$. Then the correponding piecewise polynomial (pp) function of order $k$ is defined as follows:

\[
f\left(x\right) = P_i\left(x\right) \; \textup{if } \xi_i < x < \xi_{i+1}
\] 
\noindent
for $i=1,\dots,l$. $\left\{\xi\right\}$ are known as the breakpoints of $f$. At the interior breakpoints, $\xi_2,\dots, \xi_l$, the function value is defined by specifying $f$ to be right continuous; that is, 
\[
f\left(\xi_i\right) = f\left(\xi_i^+\right),\quad i=2,\dots,l
\]
However, in a sense, without this specification, the function has two values at any interior breakpoint: the value it gets from the polynomial piece to the left of the breakpoint, $f\left(\xi_i^-\right) = P_{i-1}\left(\xi_i\right)$, in addition to the value it gets from the polynomial piece to the right of the breakpoint, $f\left(\xi_i^+\right) = P_{i}\left(\xi_i\right)$. To properly define the function, one can specify $f$ to be right-continuous:
\begin{equation}
f\left(\xi_i\right) \equiv f\left(\xi_i^+\right) 
\end{equation}

Denote the set of pp functions of order $k$ with breakpoints $\xi=\left\{\xi_1,\dots,\xi_{l+1}\right\}$ by 
\[
\mathcal{P}_{k,\xi}.
\]

$\mathcal{P}_{k,\xi}$ is a linear space having dimension $kl$, as it consists of $l$ polynomials, each having $k$ polynomial coefficients. The $j^{th}$ derivative of a pp $f$,
\[
D^jf
\]
\noindent
is a pp function of order $k-j$ having the same breakpoint sequence and constructed from the same $j^{th}$ derivatives of the polynomial pieces from which $f$ was constructed. This ``definition'' dodges much of the complicated discussion of the derivatives of a pp function at its breakpoints and thus must be treated with considerable care in context of the fundamental theorem of calculus.

\begin{proposition} \label{proposition:continuous_function}
A pp function, $f$ satisfies
\[
f\left(x\right) - f\left(a\right) = \int_a^x \left(Df\right)\left(t\right)dt\quad \textup{for all} \quad x
\]
if and only if $f$ is a continuous function.
\end{proposition}

Consider a piecewise constant function $f$: by the previous definition, its first derivative is identically zero, and is therefore equal to the usual derivative of $f$ if and only if $f$ is constant. The following definition gives the information necessary for a convenient and efficient representation of this class of functions:

\begin{definition}\label{definition:pp_representation}
The \emph{piecewise polynomial (pp) representation} of a function $f \in \PP_{k,\xi}$ consists of 
\begin{enumerate}
\item integers $k$ and $l$, specifying the order and number of polynomial pieces, respectively,
\item a strictly increasing set of breakpoints, $\xi_1,\xi_2,\dots, \xi_{l+1}$,
\item and the set of values of the right derivatives at each of the breakpoints, 
\[
c_{ij} = D^{j}f\left(\xi_i^+\right), \quad j=1,\dots,k;\quad i=1,\dots,l 
\]
\end{enumerate}
\end{definition} 
 
 \section{The truncated power basis and the spaces $\PP_{k,\xi,\nu}$}
This prerequisite information is merely for the ability to responsibly refer to the set of piecewise polynomial functions and have a shorthand way of doing so. These means enable us to introduce two sets of basis functions: first, the truncated power basis, followed by B-spline basis functions. We will see that both are closely related, with the former having some properties which leave them unattractive for function approximation and thus present the construction of B-splines and how to use them to construct a representation of $\mathcal{P}_{k}$. In practice, one typically is given some information about an unknown function, $g$, and the task is to construct a function $f \in \PP_{k, \xi}$ which satisfies conditions that $g$ also satisfies, and in addition, has a certain number of continuous derivatives. These conditions define a subspace of $\PP_{k,\xi}$, $\PP_{k,\xi, \nu}$ for which we will need a corresponding basis.


\subsection{Example: histogram smoothing by parabolic splines}
For illustrative purposes, consider the task of smoothing a histogram using parabolic splines. Suppose we are given points
\[
\tau_1 < \tau_2 < \dots < \tau_{n+1}
\]
and non-negative numbers $h_1, h_2, \dots, h_n$, with $h_i$ denoting the height of the histogram over the interval $\left(\tau_i, \tau_{i+1} \right)$. The histogram is an approximate representation of some underlying density function, $g$. Letting $\Delta \tau_i = \tau_{i+1}-\tau_i$, one may interpret $h_i\Delta \tau_i$ as (approximately) equal to the integral of $g$ over $\left[\tau_i, \tau_{i+1} \right]$. One may impose the following interpolation conditions on our smooth function, $f$:
\begin{equation*} 
\int_{\tau_i}^{\tau_{i+1}} f\left(x\right)dx = h_i\Delta \tau_i
\end{equation*} 
\noindent
for $i=1,\dots, n$. Let $f$ be a piecewise polynomial of order 3 having continuous first derivative:
\[
f \in \PP_{3,\xi} \cap \mathcal{C}^{\left(1\right)}
\]
Choose the breakpoint sequence $\xi$ to coincide with $\tau = \left\{\tau_1,\dots, \tau_{n+1} \right\}$. If $g$ is smooth and vanishes outside its support, $\left[ \tau_1,\tau_{n+1} \right]$, then
\[
g^{\left( j \right)}\left(\tau_1\right) = g^{\left( j \right)}\left(\tau_{n+1}\right) = 0,
\]
\noindent
for $j=0,1,\dots,d$, where $d$ characterizes the extent of the smoothness of $g$, we may also wish to require $f$ to obey two additional interpolation constraints:
\[
f\left( \tau_1 \right) = f\left( \tau_{n+1} \right) = 0,
\]
giving a total of $n+2$ interpolation conditions. These, along with the $2\left(n-1\right)$ continuity conditions yield a total $3n$ constraints on the $3n$ polynomial coefficients,
\[
c_{ji} \equiv D^{j-1} f\left(\xi_i^+ \right).
\]
These conditions lead to the system of equations:

\begin{align*}
c_{11} &  &  &  &  &  &  & & & & & & & &&&&= & 0  \\
c_{11} & \;+\;  & c_{21} \frac{\Delta\tau_{1}}{2!} & \;+\;   & c_{31} \frac{\left(\Delta\tau_{1}\right)^2} {3!}  & 			   & 		 		        &  	      & & & && & &&&& = & h_1\\
c_{11} & \; +\; & c_{21}\Delta\tau_{1}                 & \; + \; & c_{31}\frac{2\left(\Delta\tau_{1}\right)^2}{3!}  & \;-\; & c_{12} &  		        & 	      & & & & & &&&& = & 0\\
\vdots   &  		   & c_{21}  				     & \; + \; & c_{31} \Delta\tau_{1}  				     &  		   &  		 & \;-\;  & c_{22} & && & & &&&& = & 0\\
	   &  			   &             				     &                        & 								     &  		   &   c_{12} &\;+\; &  c_{22}\frac{\Delta\tau_2}{2} &\;+\; & c_{32}\frac{\left(\Delta\tau_2\right)^2}{3!} & & & & &&& = & h_2\\
	   &  			   &             				     &                        & 								     &  		   &   c_{12} &\;+\; &  c_{22}\Delta\tau_2        &\;+\; & c_{32}\frac{\left(\Delta\tau_2\right)^2}{2} & & &\dots &&&&  = & 0\\
	   	   &  			   &             				     &                        & 								     &  		   &   		 &	&  c_{22}        &\;+\; & c_{32}\Delta\tau_2 & & & \dots &&&&  = & 0\\
    &  &  &  &  &  &  & & & & && & &&&&  &\\
   &  &  &  &  &  &  & & & & && & &\ddots &&&  & \numberthis  \label{eq:histogram_smoothing_eqn_system}
\end{align*}

\subsection{The subspace, $\PP_{k,\xi,\nu}$}
One may quickly see that this system is two-thirds homogeneous; that is, for every integral interpolation constraint, we have two continuity constraints that lead to zeros on the right hand side of the equality. To solve \ref{eq:histogram_smoothing_eqn_system}, the homogeneous equations are solved, leaving a reduced set of equations. To do this, one may construct a set of linearly independent functions $\phi_1, \phi_2, \dots$ of the same size as the number of interpolation constraints which satisfy the homogeneous equations in \ref{eq:histogram_smoothing_eqn_system}. The smoother, $f$, is then constructed within this subspace of $\PP_{3,\xi}$ and has form 
\[
f = \sum_{j} \alpha_j \phi_j.
\] 

\noindent
The $\left\{ \phi_j \right\}$ span a particular subspace of $\mathcal{P}_{k, \xi}$, which is comprised of functions which satisfy the homogeneous equations in \ref{eq:histogram_smoothing_eqn_system}. In general, we may characterize these homogeneous equations in terms of a given function having a particular number of continuous derivatives, which may be expressed as follows:
\begin{eqnarray}
\mathscr{J}_{ij}f = 0, \quad  i&=&2,\dots,l \label{eq:tpf_linear_operator_penalty} \\
 j&=&1,\dots,\nu_i \nonumber
\end{eqnarray}
\noindent
where
\begin{equation}
\mathscr{J}_{ij}f = \lim_{x\rightarrow \xi^+} D^{j-1}f\left(x\right) - \lim_{x\rightarrow \xi^-} D^{j-1}f\left(x\right)
\end{equation}
\noindent
and $\nu = \left(\nu_1,\dots, \nu_l\right)$ is a vector of non-negative integers. Each $\nu_i$ specifies the number of continuity conditions impose on the function at the $i^{th}$ breakpoint, $\xi$, and $\mathscr{J}_{ij}$ is simply the size of the jump in the $\left(j-1\right)^{st}$ derivative at $\xi$.


\subsection{The truncated power functions}
As homogeneous conditions specified in \ref{eq:tpf_linear_operator_penalty} are done so in terms of a linear operator applied to the functions in the space, those functions $\left\{ g \in \PP_{k,\xi} \right\}$ constitute a linear subspace, denoted
\[
 \PP_{k,\xi, \nu} 
\]
\noindent
Now that the subspace is defined, we need a basis for it. One such candidate set of basis functions is the truncated power basis. Define

\[
\left(x-t\right)_+ = \max\left(0,x-t\right)
\]
\noindent
We may then define the truncated power function as follows:
\begin{align}
\begin{split}
\left(x-t\right)_+^p &= \left( \left(x-t\right)_+\right)^p\\
&= \left\{\begin{array}{lr}
	\left(x-t\right)^p, & x \ge t\\
	0 & x < t
	\end{array} \right.
	\end{split}
\end{align}

The function $g\left(x\right) = \left(\right)_+^p$ is a piecewise polynomial with a single breakpoint at $\xi$, and is continuous at this breakpoint for $p>0$. For $p=0$, it has a jump of size 1 at $\xi$. Since

\[
D\left( \cdot - \xi\right)_+^p = p \left( \cdot - \xi\right)_+^{p-1},
\]
\noindent
it is clear that $g$ has $p-1$ continuous derivatives. Define the set of linear operators $\left\{ \lambda_{ij} \right\}$ and corresponding functions $\left\{ \phi_{ij} \right\}$ as follows:

\begin{align}
\begin{split}
\lambda_{ij}f  &= \left\{\begin{array}{lr}
	 D^{j}f\left(\xi \right) & i=1\\
	D^{j}f\left(\xi^+ \right) -  D^{j}f\left(\xi^-\right) & i=2,\dots,l
	\end{array} \right. \\
\phi_{ij}\left(x\right) &= \left\{\begin{array}{lr}
	\frac{\left(x-\xi_i \right)^p}{j!} & i=1\\
	\frac{\left(x-\xi_i \right)_+^p}{j!} & i=2,\dots,l
	\end{array} \right. 	
	\end{split} \label{eq:PP_basis_decomposition_componenents}
\end{align}
\noindent for $j=0,\dots,k-1$. Per this definition, we have that $\phi_{ij} \in \PP_{k,\xi}$ for each $i=1,\dots,l$. Further, we have that 

\begin{align*}
\lambda_{ij}\phi_{pq} = \left\{\begin{array}{lr}
	1  & i=p,\;j=q\\
	0 & \textup{otherwise}
	\end{array} \right. ,
\end{align*}
\noindent
making $\left\{ \phi_{ij} \right\}$ a set of $kl$ linearly independent functions, and since $\PP_{k,\xi}$ has dimension $kl$, they constitute a basis for the space. We may represent any $g \in \PP_{k,\xi}$ in the form

\begin{equation*}
g = \sum_{i,j} \left(\lambda_{ij}g\right)\phi_{ij}
\end{equation*}
\noindent
Rewriting this expansion in the terms presented in \ref{eq:PP_basis_decomposition_componenents}, we have that we may express any function in the space as
\begin{equation} \label{eq:tpf_basis_expansion}
g\left(x\right) = \sum_{j=0}^{k-1} \left[ D^j g\left(\xi_1\right)\frac{\left(x-\xi_1\right)^j}{j!} + \sum_{i=2}^l \left[ D^j g\left(\xi_i^+\right)-D^j g\left(\xi_i^-\right) \right]\frac{\left(x-\xi_i\right)_+^j}{j!} \right]
\end{equation}

From this representation of the function, one can see that the coefficients of the basis functions are explicitly defined in terms of jumps of various derivatives of $g$ at the breakpoints. Thus, enforcing the homogeneous constraints 
\[
\mathscr{J}_{ij}f = 0, \quad i=2,\dots, l;\quad j=1,\dots,\nu_i
\]
is accomplished simply by restricting our attention to functions of the form \ref{eq:tpf_basis_expansion} for which these coefficients are zero. This implies that the reduced set of basis functions $\left\{ \phi_{ij} \right\}$, $i=1,\dots,l$, $j=\nu_{i},\dots,k-1$ is a basis for the subspace, $\PP_{k,\xi,\nu}$. (We let $\nu_1 =0$.) Then, every function $g^* \in \PP_{k,\xi}$ in satisfying the homogeneous equations may be written in exactly one way of the form
\begin{equation} \label{eq:tpf_subspace_basis_expansion}
g^* = \sum_{i=1}^l \sum_{j={\nu_i}}^{k-1} \alpha_{ij} \phi_{ij}
\end{equation}

\subsubsection{The pitfalls of the truncated power basis}

While the truncated power basis initially appears quite attractive for smoothing problems involving piecewise polynomials, they exhibit properties that can lead to poor function representations. In order to discuss these characteristics, we must first introduce the notion of the \emph{condition} of a function representation.

\begin{definition}\label{definition:basis_condition}
Consider a piecewise polynomial representation of a function, $p$:
\begin{equation} \label{eq:polynomial_representation}
p = \sum_{i=1}^n a_i P_i
\end{equation}
\noindent
where $a = \left(a_1,\dots,a_n\right)$ is a coefficient vector and $\left\{ P_i\right\}$ is a set of polynomial functions defined on a closed interval $\left[a,b\right]$. We define the \emph{size} of the polynomial $p$ to be
\[
\left \lVert p \right \rVert = \max_{a\le x\le b} \left \lvert p\left(x\right) \right \rvert,
\]
\noindent
and similarly, we define the size of the coefficient vector $a$ to be
\[
\left \lVert a \right \rVert = \max_{1\le i \le n} \left \lvert a_i \right \rvert,
\]
Then, we can bound the size of the function $p$:
\begin{equation} \label{eq:polynomial_expansion_size_bounds}
m\left \lVert a \right \rVert \le \left \lVert \sum_{i=1}^n a_i P_i \right \rVert \le  M \left \lVert a \right \rVert
\end{equation}
\noindent
where 
\begin{align*}
m &= \min_a \frac{\left \lVert \sum_{i=1}^n a_i P_i \right \rVert}{\left \lVert a \right \rVert},\quad \textup{and}\\
M &= \min_a \frac{\left \lVert \sum_{i=1}^n a_i P_i \right \rVert}{\left \lVert a \right \rVert}
\end{align*}
The \emph{condition} of the representation of $p$ as written in \ref{eq:polynomial_representation} is given by 
\[
\textit{condition}\left( P_i \right) = \frac{M}{m}
\]
\end{definition}

The condition of a function representation quantifies the extent to which a slight change in the coefficient vector will impact the function itself. To see this, rather than the coefficient vector $a$, consider instead a perturbation of $a$:
\[
a + \delta a
\]
\noindent
The corresponding perturbed polynomial is given by 
\[
p + \delta p = \sum_{i=1}^n \left(a_i + \delta a_i \right) P_i
\]
By \ref{eq:polynomial_expansion_size_bounds}, we then have that 
\begin{equation*}
\frac{m\left \lVert \delta a \right \rVert}{M\left \lVert a \right \rVert} \le \frac{\left \lVert \delta p \right \rVert}{\left \lVert p \right \rVert} \le \frac{M\left \lVert \delta a \right \rVert}{m\left \lVert a \right \rVert},
\end{equation*}
\noindent
implying that a relative change of $\frac{\left \lVert \delta a \right \rVert}{\left \lVert a \right \rVert}$ to the coefficient vector may result in a relative change to the function $p$ as large as $\frac{M}{m} = \textit{condition}\left(P_i\right)$ times the relative change in $a$ (and at least as large as $\textit{condition}\left(P_i\right)^{-1}$). Note that the width of the interval 
\[
\left[\textit{condition}\left(P_i\right)^{-1}\frac{\left \lVert \delta a \right \rVert}{\left \lVert a \right \rVert},\textit{condition}\left(P_i\right) \frac{\left \lVert \delta a \right \rVert}{\left \lVert a \right \rVert} \right]
\]
\noindent
is increasing in $\textit{condition}\left(P_i\right)$; so large values of the condition of a representation imply that small relative changes in their corresponding coefficients may result in much smaller or much larger relative changes in the function being represented.

When representing a function $f \in \PP_{k,\xi}$ as in \ref{eq:tpf_basis_expansion}, two issues of concern arise: first, if $l$ is large, the value of the function at a point $x$ can potentially rely on far more than just $k$ of the basis coefficients. Additionally, if the breakpoints $\xi$ are very irregularly spaced, then the truncated power basis can present poor condition, which, in turn, can result in poorly conditioned linear systems (like the specific example given by \ref{eq:histogram_smoothing_eqn_system}.) Consequently, small perturbations of the basis function coefficients result in disproportional changes in the function.
\begin{figure}[H] 
\begin{center}
\begin{tikzpicture}
\draw[->] (-1,0) -- (11,0);
\draw[->] (0,-1) -- (0,3.5);
\filldraw[black] (0.1,1.3) circle (2pt)  ;
\filldraw[black] (1.1,1.3) circle (2pt)  ;
\filldraw[black] (2.1,1.3) circle (2pt)  ;
\filldraw[black] (3.1,1.3) circle (2pt)  ;
\filldraw[black] (4.45,2.4) circle (2pt)  ;
\filldraw[black] (4.75,0.2) circle (2pt)  ;
\filldraw[black] (6.1,1.3) circle (2pt)  ;
\filldraw[black] (7.1,1.3) circle (2pt)  ;
\filldraw[black] (8.,1.3) circle (2pt)  ;
\filldraw[black] (9.1,1.3) circle (2pt)  ;
\filldraw[black] (10,1.3) circle (2pt)  ;

\draw (0.1,1.3) -- (3.1,1.3);
\draw (3.1,1.3) -- (4.45,2.4);
\draw (4.45,2.4) -- (4.75,0.2);
\draw (4.75,0.2) -- (6.1,1.3);
\draw (6.1,1.3) -- (10,1.3);
\draw (0.1,0) node[above,scale=0.6] {$\xi_{1}$};
\draw (1.1,0) node[above,scale=0.6] {$\xi_{2}$};
\draw (2.1,0) node[above,scale=0.6] {$\xi_{3}$};
\draw (3.1,0) node[above,scale=0.6] {$\xi_{4}$};

 \draw (4.45,-.38) -- (4.45,-.22); 
 \draw (4.75,-.38) -- (4.75,-.22); 
\draw (4.45,-.3) -- (4.75,-.3) ;
\draw (4.6,-0.3) node[below,scale=0.6,align=center] {h};

\draw (6.1,0) node[above,scale=0.6] {$\xi_{7}$};
\draw (7.1,0) node[above,scale=0.6] {$\xi_{8}$};
\draw (8.1,0) node[above,scale=0.6] {$\xi_{9}$};
\draw (9.1,0) node[above,scale=0.6] {$\xi_{10}$};
\draw (10,0) node[above,scale=0.6] {$\xi_{11}$};
\foreach \coo in {.1,1.1,...,3.1}
{
  \draw (\coo,-1.5pt) -- (\coo,1.5pt); 
  \draw (\coo,0) node[below,scale=0.6] {\coo};
}
 \draw (4.45,-1.5pt) -- (4.45,1.5pt);
  \draw (4.75,-1.5pt) -- (4.75,1.5pt);
\foreach \coo in {6.1,7.1,...,9.1}
{
  \draw (\coo,-1.5pt) -- (\coo,1.5pt);
  \draw (\coo,0) node[below,scale=0.65] {\coo};
}
 \draw (10,0) node[below,scale=0.65] {10};
\draw (10,-1.5pt) -- (10,1.5pt);
\draw (-1.5pt,0.2) -- (1.5pt,0.2);
\draw (-1.5pt,1.3) -- (1.5pt,1.3);
\draw (-1.5pt,2.4) -- (1.5pt,2.4);
\draw (0,0.2) node[left,scale=0.6] {0.2};
\draw (0,1.3) node[left,scale=0.6] {1.3};
\draw (0,2.4) node[left,scale=0.6] {2.4};
\end{tikzpicture} 
\caption{The linear system for the coefficients of the truncated power basis is ill-conditioned for the above piecewise linear function  and choice of breakpoints, $\xi_1,\dots, \xi_{11}$.} \label{fig:piecewise_linear_tpf_fail}
\end{center}
\end{figure}


\begin{example} \label{ex:tpf_fail}
Suppose we wish to construct a function $f \in \PP_{k=2,\xi} \cap \mathscr{C}^{\left(0\right)}$ satisfying $f\left(\xi_i \right)=1.3$ for $i=1,\dots,4,7,\dots,11$, $f\left(\xi_5\right)=2.4$, and $f\left(\xi_6\right)=0.2$ with breakpoints specified as pictured in Fig~\ref{fig:piecewise_linear_tpf_fail}. 


The function $f$ is piecewise linear, so that $f \in \PP_{2,\xi,\nu}$ with $\nu_i=1$, $i=1,\dots, 11$. Then, for $f$ is of the form
\[
f\left(x\right) = \alpha + \beta\left(x-\xi_1\right) + \sum_{i=2}^l\alpha_i\left(x-\xi_i\right)_+
\]
\noindent

one can show that, given the continuity and interpolation constraints, $\alpha = 1.3$, $\beta = f^\prime\left(\xi_1\right) = 0$,  $\alpha_i = 0$ for $i \not \in \left\{4,5,6,7\right\}$, and 
\[
\begin{bmatrix}
\alpha_4\\
\alpha_5\\
\alpha_6\\
\alpha_7\\
\end{bmatrix} =  \begin{bmatrix}
\frac{1.1}{\Delta\xi_4}\\
-\frac{2.2}{\Delta \xi_5} -\frac{1.1}{\Delta\xi_4}\\
\frac{1.1}{\Delta\xi_6} + \frac{2.2}{\Delta \xi_5}\\
-\frac{1.1}{\Delta\xi_6} 
\end{bmatrix} = \begin{bmatrix}
\frac{1.1}{\Delta\xi_4}\\
-\frac{2.2}{h} -\frac{1.1}{\Delta\xi_4}\\
\frac{1.1}{\Delta\xi_6} + \frac{2.2}{h}\\
-\frac{1.1}{\Delta\xi_6} 
\end{bmatrix}
\]

When $h = \Delta \xi_5 \rightarrow 0$, we have 
\begin{align*}
\left(x-\xi_5\right)_+ &\approx \left(x-\xi_6\right)_+\\
\alpha_5  &\approx -\alpha_6 >> 1,
\end{align*}
\noindent
leading to a loss of significance in the evaluation of the function. For example, if we were to choose $\xi_5=4.5$, $\xi_6=4.8$, making $h=0.3$, with five significant decimal digit arithmetic, then we get
\[
\alpha=1.30000,\; \beta= 0.00000,\;\alpha_4= 0.78571,\;\alpha_5= -8.11905,\;\alpha_6= 8.17949,\;\alpha_7= -0.84615
\]
and the other $\alpha_i=0$. Evaluating the corresponding function at $x=9.6$ yields $1.299987$ rather than the correct value of 1.3. This error becomes larger as $h$ approaches 0.
\end{example}

A quick remedy for the problem in \ref{ex:tpf_fail} is to replace the truncated power basis with the set of hat functions:

\[
H_i\left(x\right)=\left\{
\begin{array}{lr}
\left(x-\xi_i\right)/\Delta\xi_{i-1}, & \xi_{i-1} < x \le \xi_i\\
\left(\xi_{i+1}-x\right)/\Delta\xi_{i}, & \xi_{i} < x \le \xi_{i+1}\\
0 & otherwise
\end{array}  \right.
\]
\noindent
To utilize the hat functions, we augment $\xi_1,\dots,\xi_12$ with two additional breakpoints: $\xi_0 \le \xi_1$ and $\xi_{12} \ge \xi_{11}$. Then,

\begin{align*}
f\left(x\right) = 1.3H_1\left(x\right) + \dots + 1.3H_4\left(x\right) &+ 2.4H_5\left(x\right) \\
 &+   0.2H_6\left(x\right)+ 1.3H_7\left(x\right) + \dots + 1.3H_{11}\left(x\right)
\end{align*}

Even just using just two decimal digit arithmetic, we have $f\left(9.5\right) = 1.3$. Even as $h \rightarrow 0$, $f$ is well represented using the hat functions as a basis. More generally speaking, \emph{B-splines}, which are a generalization of these hat functions, overcome the issues that accompany the truncated power basis previously illustrated. The alternative basis is constructed by assembling linear combinations of the truncated power functions, forming a set of basis functions having ``small'' support: the functions vanish outside a small interval of their domain. In the following section, we will define the $k^{th}$-order B-splines as scaled \emph{divided differences} of truncated power functions. We will also show that every subspace $\PP_{k,\xi,\nu}$ has a basis consisting of these functions, bringing forth a B-spline representation of any pp function. 

\section{The B-spline representation of piecewise polynomial functions}

We will present an introduction to B-splines and their properties in the section to follow. As these basis functions are defined in terms of divided differences of the truncated power functions discussed in the previous section, we must first review the definition and properties of the divided difference operator. While there are many ways of defining divided differences, the following (somewhat nonconstructive definition) is intuitive and adequate for our purposes here.

\subsection{The divided difference}

\begin{definition}\label{definition:divided_difference}
The \emph{$k^{th}$ divided difference} of a function $g$ at the points $\tau_i,\dots,\tau_{i+k}$, denoted
\[
\left[ \tau_i,\dots,\tau_{i+k} \right]g,
\]
is the leading coefficient of the polynomial of order $k+1$ which \emph{agrees with $g$} at $\tau_i,\dots,\tau_{i+k}$.
\end{definition}
This leads us to the following definition:

\begin{definition}\label{definition:agrees_with_g}
Let $\left\{\tau_i \right\}_{i=1}^n$ denote a sequence of points. We say that a function $f$ \emph{agrees with} a function $g$ at $\tau$ if, for every point $\tau$ which occurs $m$ times in $\left\{\tau_i \right\}$, $f$ and $g$ agree up to $m$ derivatives at $\tau$; i.e.
\[
f^{\left(i-1\right)}\left(\tau\right) = g^{\left(i-1\right)}\left(\tau\right), \quad i=1,\dots,m
\]
\end{definition}

\subsubsection{Properties of the divided difference}
Definition~\ref{definition:divided_difference} yields the following properties of the $k^{th}$ divided difference:

\begin{enumerate} 
\item Let $\left\{ p_i \right\}$, $i=1,2,\dots$ denote a sequence of polynomials with $p_i$ having order $i$. If $p_i$ agrees with a function $g$ at $\tau_1,\dots, \tau_i$ for $i=k,k+1$, then 
	\begin{equation*} 
	p_{k+1}\left(x\right) = p_k\left(x\right) + \left(x-\tau_1\right)\cdot \dots \cdot \left(x-\tau_k\right)\left[\tau_1,\dots,\tau_{k+1}\right]g
	\end{equation*}
	\begin{proof}
	Note that $p_{k+1}\left(x\right) - p_{k}\left(x\right)$ is a polynomial of order $k+1$ and vanishes at $\tau_1,\dots,\tau_k$, and, by definition, has $\left[\tau_1,\dots,\tau_{k+1}\right]g$ as its leading coefficient. Consequently, $p_{k+1}-p_k$ is of the form
	\begin{equation*}
	p_{k+1}\left(x\right) - p_{k}\left(x\right) = c\prod_{j=1}^k \left(x-\tau_j\right)
	\end{equation*}
	\noindent
	where 
	\[
	c = \left[\tau_1,\dots,\tau_{k+1}\right]g
	\]
	\end{proof}
	This tells the reader that interpolating polynomials may be constructed using divided differences by adding the interpolation points one by one, giving the \emph{Newton form} of the $n^{th}$ order polynomial which agrees with $g$ at $\tau_1,\dots,\tau_n$:
	\[
	p_n\left(x\right) = \sum_{i=1}^n \left(x-\tau_{1}\right)\times \dots \times \left(x-\tau_{i-1}\right) \left[\tau_1,\dots,\tau_{i-1} \right]g
	\]
\item $\left[\tau_i,\dots,\tau_{i+k} \right]g$	is symmetric in its arguments $\tau_i,\dots, \tau_{i+k}$ (since the interpolating polynomial depends only on the points of interpolation and not the order in which they are specified.)
\item $\left[\tau_i,\dots,\tau_{i+k} \right]g$ is a linear operator; if we let 
\[
f= \alpha g + \beta h
\]
\noindent
for some functions $g,h$ and scalars $\alpha, \beta$, then
\[
\left[\tau_i,\dots,\tau_{i+k} \right]f = \alpha\left[\tau_i,\dots,\tau_{i+k} \right]g + \beta\left[\tau_i,\dots,\tau_{i+k} \right]h
\]
\item If $f=gh$, then
\[
\left[\tau_i,\dots,\tau_{i+k} \right]f  = \sum_{j=i}^{i+k} \left[\tau_i,\dots,\tau_{i+k} \right]g \left[\tau_i,\dots,\tau_{i+k} \right]h 
\]
\item $\left[\tau_i,\dots,\tau_{i+k} \right]g$ is a continuous function of its $k+1$ arguments in case $g\in \mathscr{C}^{\left(k\right)}$ (that is, if $g$ has $k$ continuous derivatives.)
\item If $g\in \mathscr{C}^{\left(k\right)}$, then there exists a point $\xi$ in the smallest interval containing $\tau_i,\dots,\tau_{i+k}$ such that 
\[
\left[\tau_i,\dots,\tau_{i+k} \right]g = \frac{g^{\left(k\right)}\left(\xi\right)}{k!}
\] 
\item \[
\left[\tau_i,\dots,\tau_{i+k} \right]g = \left\{\begin{array}{ll}
\frac{g^{\left(k\right)}\left(\tau_i\right)}{k!} & \tau_i= \dots = \tau_{i+k},\;\;g\in \mathscr{C}^{\left(k\right)} \\
\frac{\left[\tau_i,\dots,\tau_{r-1},\tau_{r+1},\dots,\tau_{i+k} \right]g-\left[\tau_i,\dots,\tau_{s-1},\tau_{s+1},\dots,\tau_{i+k} \right]g}{\tau_s-\tau_r} &  \tau_r,\tau_s \textup{ are any two distinct}\\
& \textup{  points in } \left\{\tau_i,\dots,\tau_{i+k}\right\}
\end{array} \right.
\]
\end{enumerate}

In the following section, we will present B-splines as divided differences of the truncated power basis and present some of their properties which come as a result of the properties of the divided difference presented in this section.

\section{A B-spline representation for pp functions}

\begin{definition} \label{definition:order_k_Bspline}
Let $t= \left\{ t_i \right\}$ denote a non-decreasing sequence. The $i^{th}$ B-spline of order $k$ which corresponds to the knot sequence $t$ is defined by 
\begin{equation} \label{eq:bspline_definition}
B_{i,k,t}\left(x\right) = \left(t_{i+k}-t_i\right)\left[t_i,\dots,t_{i+k}\right]\left(\cdot -x\right)_+^{k-1}
\end{equation}
\end{definition}

The placeholder notation, $\left(\cdot - x\right)_+^{k-1}$, 

\subsection{Properties of B-splines}

\begin{enumerate}
\item Let 
\end{enumerate}

\end{document}