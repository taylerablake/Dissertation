\documentclass[12pt]{article}
\usepackage{graphicx,psfrag,amsfonts,float,mathbbol,xcolor,cleveref}
\usepackage{arydshln}
\usepackage{amsmath}
\usepackage{subcaption}
\usepackage{mathtools}
\usepackage{IEEEtrantools}
\usepackage[letterpaper, left=1in, top=1in, right=1in, bottom=1in,nohead,includefoot, verbose, ignoremp]{geometry}
\newcommand*\needsparaphrased{\color{red}}
\newcommand*\needsproof{\color{blue}}
\newcommand*\outlineskeleton{\color{green}}
\newcommand{\bfeps}{\mbox{\boldmath $\epsilon$}}
\newcommand{\bfgamma}{\mbox{\boldmath $\gamma$}}
\newcommand{\bflam}{\mbox{\boldmath $\lambda$}}
\newcommand{\bfphi}{\mbox{\boldmath $\phi$}}
\newcommand{\bfsigma}{\mbox{\boldmath $\sigma$}}
\newcommand{\bfbeta}{\mbox{\boldmath $\beta$}}
\newcommand{\bfalpha}{\mbox{\boldmath $\alpha$}}
\newcommand{\bfe}{\mbox{\boldmath $e$}}
\newcommand{\bff}{\mbox{\boldmath $f$}}
\newcommand{\bfone}{\mbox{\boldmath $1$}}
\newcommand{\bft}{\mbox{\boldmath $t$}}
\newcommand{\bfo}{\mbox{\boldmath $0$}}
\newcommand{\bfO}{\mbox{\boldmath $O$}}
\newcommand{\bfx}{\mbox{\boldmath $x$}}
\newcommand{\bfX}{\mbox{\boldmath $X$}}
\newcommand{\bfz}{\mbox{\boldmath $z$}}


\newcommand{\bfm}{\mbox{\boldmath $m}}
\newcommand{\bfy}{\mbox{\boldmath $y$}}
\newcommand{\bfd}{\mbox{\boldmath $d$}}
\newcommand{\bfc}{\mbox{\boldmath $c$}}
\newcommand{\bfa}{\mbox{\boldmath $a$}}
\newcommand{\bfb}{\mbox{\boldmath $b$}}
\newcommand{\bfY}{\mbox{\boldmath $Y$}}
\newcommand{\bfS}{\mbox{\boldmath $S$}}
\newcommand{\bfZ}{\mbox{\boldmath $Z$}}
\newcommand{\cardT}{\vert \mathcal{T} \vert}
\newenvironment{theorem}[1][Theorem]{\begin{trivlist}
\item[\hskip \labelsep {\bfseries #1}]}{\end{trivlist}}
\newenvironment{corollary}[1][Corollary]{\begin{trivlist}
\item[\hskip \labelsep {\bfseries #1}]}{\end{trivlist}}
\newenvironment{proposition}[1][Proposition]{\begin{trivlist}
\item[\hskip \labelsep {\bfseries #1}]}{\end{trivlist}}
\newenvironment{definition}[1][Definition]{\begin{trivlist}
\item[\hskip \labelsep {\bfseries #1}]}{\end{trivlist}}

\def\bL{\mathbf{L}}


 \begin{document}

\nocite{*}

\section{{\outlineskeleton The Truncated Power Basis}}
\subsection{{\outlineskeleton Piecewise Polynomial Functions}}

Let $\xi = \left\{ \xi_1<\xi_2<\dots<\xi_{l+1} \right\}$ be a set of strictly increasing series of points, and let $k$ be a positive integer. Further, let $P_1,\dots,P_l$ denote a sequence of $l$ polynomials of order $k$. Then the correponding piecewise polynomial (pp) function of order $k$ is defined as follows:

\[
f\left(x\right) = P_i\left(x\right) \quad \textup{if } \xi_i < x < \xi_{i+1}
\] 
\noindent
for $i=1,\dots,l$. $\left\{\xi\right\}$ are known as the breakpoints of $f$. At the interior breakpoints, $\xi_2,\dots, \xi_l$, the function value is defined by specifying $f$ to be right continuous; that is, 
\[
f\left(\xi_i\right) = f\left(\xi_i^+\right),\quad i=2,\dots,l
\]
However, in a sense, without this specification, the function has two values at any interior breakpoint: the value it gets from the polynomial piece to the left of the breakpoint, $f\left(\xi_i^-\right) = P_{i-1}\left(\xi_i\right_$, in addition to the value it gets from the polynomial piece to the right of the breakpoint, $f\left(\xi_i^+\right) = P_{i}\left(\xi_i\right)$. To properly define the function, one can specify $f$ to be right-continuous:
\begin{equation}
f\left(\xi_i\right) \def f\left(\xi_i^+\right) 
\end{equation}

Denote the set of pp functions of order $k$ with breakpoints $\xi=\left\{\xi_1,\dots,\xi_{l+1}\right\}$ by 
\[
\mathcal{P}_{k,\xi}.
\]

$\mathcal{P}_{k,\xi}$ is a linear space having dimension $kl$, as it consists of $l$ polynomials, each having $k$ polynomial coefficients. The $j^{th}$ derivative of a pp $f$,
\[
D^jf
\]
\noindent
is a pp function of order $k-j$ having the same breakpoint sequence and constructed from the same $j^{th}$ derivatives of the polynomial pieces from which $f$ was constructed. In practice, one typically is given some information about an unknown function, $g$, and













\end{document}