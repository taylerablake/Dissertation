\documentclass[12pt]{article}
\usepackage{graphicx,psfrag,amsfonts,float,mathbbol,xcolor,cleveref}
\usepackage{arydshln}
\usepackage{amsmath}
\usepackage{subcaption}
\usepackage{mathtools}
\usepackage{IEEEtrantools}
\usepackage[letterpaper, left=1in, top=1in, right=1in, bottom=1in,nohead,includefoot, verbose, ignoremp]{geometry}
\newcommand{\comment}[1]{\text{\phantom{(#1)}} \tag{#1}}
\newcommand\numberthis{\addtocounter{equation}{1}\tag{\theequation}}
\newcommand*\needsparaphrased{\color{red}}
\newcommand*\needsproof{\color{blue}}
\newcommand*\outlineskeleton{\color{green}}
\newcommand{\PP}{\mathcal{P}}
\newcommand{\bfeps}{\mbox{\boldmath $\epsilon$}}
\newcommand{\bfgamma}{\mbox{\boldmath $\gamma$}}
\newcommand{\bflam}{\mbox{\boldmath $\lambda$}}
\newcommand{\bfphi}{\mbox{\boldmath $\phi$}}
\newcommand{\bfsigma}{\mbox{\boldmath $\sigma$}}
\newcommand{\bfbeta}{\mbox{\boldmath $\beta$}}
\newcommand{\bfalpha}{\mbox{\boldmath $\alpha$}}
\newcommand{\bfe}{\mbox{\boldmath $e$}}
\newcommand{\bff}{\mbox{\boldmath $f$}}
\newcommand{\bfone}{\mbox{\boldmath $1$}}
\newcommand{\bft}{\mbox{\boldmath $t$}}
\newcommand{\bfo}{\mbox{\boldmath $0$}}
\newcommand{\bfO}{\mbox{\boldmath $O$}}
\newcommand{\bfx}{\mbox{\boldmath $x$}}
\newcommand{\bfX}{\mbox{\boldmath $X$}}
\newcommand{\bfz}{\mbox{\boldmath $z$}}


\newcommand{\bfm}{\mbox{\boldmath $m}}
\newcommand{\bfy}{\mbox{\boldmath $y$}}
\newcommand{\bfd}{\mbox{\boldmath $d$}}
\newcommand{\bfc}{\mbox{\boldmath $c$}}
\newcommand{\bfa}{\mbox{\boldmath $a$}}
\newcommand{\bfb}{\mbox{\boldmath $b$}}
\newcommand{\bfY}{\mbox{\boldmath $Y$}}
\newcommand{\bfS}{\mbox{\boldmath $S$}}
\newcommand{\bfZ}{\mbox{\boldmath $Z$}}
\newcommand{\cardT}{\vert \mathcal{T} \vert}
\newenvironment{theorem}[1][Theorem]{\begin{trivlist}
\item[\hskip \labelsep {\bfseries #1}]}{\end{trivlist}}
\newenvironment{corollary}[1][Corollary]{\begin{trivlist}
\item[\hskip \labelsep {\bfseries #1}]}{\end{trivlist}}
\newenvironment{proposition}[1][Proposition]{\begin{trivlist}
\item[\hskip \labelsep {\bfseries #1}]}{\end{trivlist}}
\newenvironment{definition}[1][Definition]{\begin{trivlist}
\item[\hskip \labelsep {\bfseries #1}]}{\end{trivlist}}

\def\bL{\mathbf{L}}


 \begin{document}

\nocite{*}

\section{{\outlineskeleton The Truncated Power Basis}}
\subsection{{\outlineskeleton Piecewise Polynomial Functions}}

Let $\xi = \left\{ \xi_1<\xi_2<\dots<\xi_{l+1} \right\}$ be a set of strictly increasing series of points, and let $k$ be a positive integer. Further, let $P_1,\dots,P_l$ denote a sequence of $l$ polynomials of order $k$. Then the correponding piecewise polynomial (pp) function of order $k$ is defined as follows:

\[
f\left(x\right) = P_i\left(x\right) \; \textup{if } \xi_i < x < \xi_{i+1}
\] 
\noindent
for $i=1,\dots,l$. $\left\{\xi\right\}$ are known as the breakpoints of $f$. At the interior breakpoints, $\xi_2,\dots, \xi_l$, the function value is defined by specifying $f$ to be right continuous; that is, 
\[
f\left(\xi_i\right) = f\left(\xi_i^+\right),\quad i=2,\dots,l
\]
However, in a sense, without this specification, the function has two values at any interior breakpoint: the value it gets from the polynomial piece to the left of the breakpoint, $f\left(\xi_i^-\right) = P_{i-1}\left(\xi_i\right)$, in addition to the value it gets from the polynomial piece to the right of the breakpoint, $f\left(\xi_i^+\right) = P_{i}\left(\xi_i\right)$. To properly define the function, one can specify $f$ to be right-continuous:
\begin{equation}
f\left(\xi_i\right) \equiv f\left(\xi_i^+\right) 
\end{equation}

Denote the set of pp functions of order $k$ with breakpoints $\xi=\left\{\xi_1,\dots,\xi_{l+1}\right\}$ by 
\[
\mathcal{P}_{k,\xi}.
\]

$\mathcal{P}_{k,\xi}$ is a linear space having dimension $kl$, as it consists of $l$ polynomials, each having $k$ polynomial coefficients. The $j^{th}$ derivative of a pp $f$,
\[
D^jf
\]
\noindent
is a pp function of order $k-j$ having the same breakpoint sequence and constructed from the same $j^{th}$ derivatives of the polynomial pieces from which $f$ was constructed. This ``definition'' dodges much of the complicated discussion of the derivatives of a pp function at its breakpoints and thus must be treated with considerable care in context of the fundamental theorem of calculus.

\begin{proposition} \label
A pp function, $f$ satisfies
\[
f\left(x\right) - f\left(a\right) = \int_a^x \left(Df\right)\left(t\right)dt\quad \textup{for all} \quad x
\]
if and only if $f$ is a continuous function.
\end{proposition}

Consider a piecewise constant function $f$: by the previous definition, its first derivative is identically zero, and is therefore equal to the usual derivative of $f$ if and only if $f$ is constant.  

This prerequisite information is merely for the ability to responsibly refer to the set of piecewise polynomial functions and have a shorthand way of doing so. These means enable us to introduce two sets of basis functions: first, the truncated power basis, followed by B-spline basis functions. We will see that both are closely related, with the former having some properties which leave them unattractive for function approximation and thus present the construction of B-splines and how to use them to construct a representation of $\mathcal{P}_{k}$. In practice, one typically is given some information about an unknown function, $g$, and the task is to construct a function $f \in \PP_{k, \xi}$ which satisfies conditions that $g$ also satisfies, and in addition, has a certain number of continuous derivatives. These conditions define a subspace of $\PP_{k,\xi}$, $\PP_{k,\xi, \nu}$ for which we will need a corresponding basis.

For illustrative purposes, consider the task of smoothing a histogram using parabolic splines. Suppose we are given points
\[
\tau_1 < \tau_2 < \dots < \tau_{n+1}
\]
and non-negative numbers $h_1, h_2, \dots, h_n$, with $h_i$ denoting the height of the histogram over the interval $\left(\tau_i, \tau_{i+1} \right)$. The histogram is an approximate representation of some underlying density function, $g$. Letting $\Delta \tau_i = \tau_{i+1}-\tau_i$, one may interpret $h_i\Delta \tau_i$ as (approximately) equal to the integral of $g$ over $\left[\tau_i, \tau_{i+1} \right]$. One may impose the following interpolation conditions on our smooth function, $f$:
\begin{equation*} 
\int_{\tau_i}^{\tau_{i+1}} f\left(x\right)dx = h_i\Delta \tau_i
\end{equation*} 
\noindent
for $i=1,\dots, n$. Let $f$ be a piecewise polynomial of order 3 having continuous first derivative:
\[
f \in \PP_{3,\xi} \cap \mathcal{C}^{\left(1\right)}
\]
Choose the breakpoint sequence $\xi$ to coincide with $\tau = \left\{\tau_1,\dots, \tau_{n+1} \right\}$. If $g$ is smooth and vanishes outside its support, $\left[ \tau_1,\tau_{n+1} \right]$, then
\[
g^{\left( j \right)}\left(\tau_1\right) = g^{\left( j \right)}\left(\tau_{n+1}\right) = 0,
\]
\noindent
for $j=0,1,\dots,d$, where $d$ characterizes the extent of the smoothness of $g$, we may also wish to require $f$ to obey two additional interpolation constraints:
\[
f\left( \tau_1 \right) = f\left( \tau_{n+1} \right) = 0,
\]
giving a total of $n+2$ interpolation conditions. These, along with the $2\left(n-1\right)$ continuity conditions yield a total $3n$ constraints on the $3n$ polynomial coefficients,
\[
c_{ji} \equiv D^{j-1} f\left(\xi_i^+ \right).
\]
These conditions lead to the system of equations:

\begin{align*}
c_{11} &  &  &  &  &  &  & & & & & & & &&&&= & 0  \\
c_{11} & \;+\;  & c_{21} \frac{\Delta\tau_{1}}{2!} & \;+\;   & c_{31} \frac{\left(\Delta\tau_{1}\right)^2} {3!}  & 			   & 		 		        &  	      & & & && & &&&& = & h_1\\
c_{11} & \; +\; & c_{21}\Delta\tau_{1}                 & \; + \; & c_{31}\frac{2\left(\Delta\tau_{1}\right)^2}{3!}  & \;-\; & c_{12} &  		        & 	      & & & & & &&&& = & 0\\
\vdots   &  		   & c_{21}  				     & \; + \; & c_{31} \Delta\tau_{1}  				     &  		   &  		 & \;-\;  & c_{22} & && & & &&&& = & 0\\
	   &  			   &             				     &                        & 								     &  		   &   c_{12} &\;+\; &  c_{22}\frac{\Delta\tau_2}{2} &\;+\; & c_{32}\frac{\left(\Delta\tau_2\right)^2}{3!} & & & & &&& = & h_2\\
	   &  			   &             				     &                        & 								     &  		   &   c_{12} &\;+\; &  c_{22}\Delta\tau_2        &\;+\; & c_{32}\frac{\left(\Delta\tau_2\right)^2}{2} & & &\dots &&&&  = & 0\\
	   	   &  			   &             				     &                        & 								     &  		   &   		 &	&  c_{22}        &\;+\; & c_{32}\Delta\tau_2 & & & \dots &&&&  = & 0\\
    &  &  &  &  &  &  & & & & && & &&&&  &\\
   &  &  &  &  &  &  & & & & && & &\ddots &&&  & \numberthis  \label{eq:histogram_smoothing_eqn_system}
\end{align*}

One may quickly see that this system is two-thirds homogeneous; that is, for every integral interpolation constraint, we have two continuity constraints that lead to zeros on the right hand side of the equality. To solve \ref{eq:histogram_smoothing_eqn_system}, the homogeneous equations are solved, leaving a reduced set of equations. To do this, one may construct a set of linearly independent functions $\phi_1, \phi_2, \dots$ of the same size as the number of interpolation constraints which satisfy the homogeneous equations. The smoother, $f$, is then constructed within this subspace of $\PP_{3,\xi}$ and has form 
\[
f = \sum_{j} \alpha_j \phi_j.
\] 

\noindent
The construction of $f$ as a linear combination of the $\left\{ \phi_j \right\}$ constitutes finding a basis for the subspace of the piecewise polynomials comprised of functions in this space which satisfy the homogeneous equations in \ref{eq:histogram_smoothing_eqn_system}. Let $\PP_{k,\xi,\nu}$ denote this subspace of $\PP_{k,\xi}$ which is comprised of functions having some number of continuous derivatives

\end{document}