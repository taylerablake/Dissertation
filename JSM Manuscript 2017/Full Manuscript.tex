%&latex
\documentclass[12pt]{article}

\usepackage[dvips]{graphicx}
\usepackage{caption}
\usepackage[authoryear]{natbib}
\usepackage{psfrag,subfigure, natbib,amsfonts,float,mathbbol,natbib,amsmath,animate,multicol}
\usepackage[letterpaper, left=1in, top=1in, right=1in, bottom=1in,nohead,includefoot, verbose, ignoremp]{geometry}
\usepackage{times}
\usepackage{relsize}
\usepackage{array}
\usepackage{arydshln}
\usepackage{lmodern}
\usepackage{color, colortbl}
%\definecolor{Gray}{gray}{0.9}
\newcommand*\needsparaphrased{\color{red}}
\usepackage{slantsc}
\newcommand{\bfeps}{\mbox{\boldmath $\epsilon$}}
\newcommand{\bfeta}{\mbox{\boldmath $\eta$}}
\newcommand{\bfgamma}{\mbox{\boldmath $\gamma$}}
\newcommand{\bfbeta}{\mbox{\boldmath $\beta$}}
\newcommand{\bfalpha}{\mbox{\boldmath $\alpha$}}
\newcommand{\bftheta}{\mbox{\boldmath $\theta$}}
\newcommand{\matDelta}{\mbox{$\Delta$}}
\newcommand{\bflam}{\mbox{\boldmath $\lambda$}}
\newcommand{\bfphi}{\mbox{\boldmath $\phi$}}
\newcommand{\bfxi}{\mbox{\boldmath $\xi$}}
\newcommand{\bfmu}{\mbox{\boldmath $\mu$}}
\newcommand{\bfsigma}{\mbox{\boldmath $\sigma$}}
\newcommand{\bfe}{\mbox{\boldmath $e$}}
%\newcommand{\bfb}{\mbox{\boldmath $b$}}
%\newcommand{\bfa}{\mbox{\boldmath $a$}}
\newcommand{\bft}{\mbox{\boldmath $t$}}
\newcommand{\bfo}{\mbox{\boldmath $0$}}
\newcommand{\bfone}{\mbox{\boldmath $1$}}
\newcommand{\bfx}{\mbox{\boldmath $x$}}
\newcommand{\bfz}{\mbox{\boldmath $z$}}
\newcommand{\lmr}{\fontfamily{lmr}\selectfont} % Latin Modern Roman
\newcommand{\lmss}{\fontfamily{lmss}\selectfont} % Latin Modern Sans
\newcommand{\lmm}{\fontfamily{lmtt}\selectfont} % Latin Modern Mono
\newcommand{\Ycov}{\mbox{Y}_{\textsc{\relsize{-2}{\textsl{(-n)}}}}}
\newcommand{\Yres}{\mbox{Y}_{\textsc{\relsize{-2}{\textsl{(-1)}}}}}

\newcommand{\bfm}{\mbox{\boldmath $m}}
%\newcommand{\bfy}{\mbox{\boldmath $y$}}
\newcommand{\bfy}{\mbox{\bf y}}
\newcommand{\bfa}{\mbox{\bf a}}
\newcommand{\bfb}{\mbox{\bf b}}
\newcommand{\bfd}{\mbox{\bf d}}
\newcommand{\bfomega}{\mbox{\boldmath $\omega$}}
\newcommand{\bfc}{\mbox{\bf c}}
%\newcommand{\bfY}{\mbox{\boldmath $Y$}}
\newcommand{\bfY}{\mbox{\bf Y}}
\newcommand{\bigY}{\mbox{Y}}
\newcommand{\matK}{\mbox{K}}
\newcommand{\matC}{\mbox{C}}
\newcommand{\matU}{\mbox{U}}
\newcommand{\matL}{\mbox{L}}
\newcommand{\matH}{\mbox{H}}
\newcommand{\matB}{\mbox{B}}
\newcommand{\matQ}{\mbox{Q}}
\newcommand{\matR}{\mbox{R}}
\newcommand{\matD}{\mbox{D}}
\newcommand{\matZ}{\mbox{Z}}
%\newcommand{\matH}{\mbox{H}}
\newcommand{\matT}{\mbox{T}}
\newcommand{\bfS}{\mbox{\boldmath $S$}}
\newcommand{\bfZ}{\mbox{\boldmath $Z$}}
\newcommand{\cardT}{\vert \mathcal{T} \vert}

%\DeclareGraphicsExtensions{.pdf,.png,.jpg}

\newenvironment{theorem}[1][Theorem]{\begin{trivlist}
\item[\hskip \labelsep {\bfseries #1}]}{\end{trivlist}}
\newenvironment{corollary}[1][Corollary]{\begin{trivlist}
\item[\hskip \labelsep {\bfseries #1}]}{\end{trivlist}}
\newenvironment{proposition}[1][Proposition]{\begin{trivlist}
\item[\hskip \labelsep {\bfseries #1}]}{\end{trivlist}}
\newenvironment{algorithm}[1][Algorithm]{\begin{trivlist}
\item[\hskip \labelsep {\bfseries #1}]}{\end{trivlist}}
\newenvironment{definition}[1][Definition]{\begin{trivlist}
\item[\hskip \labelsep {\bfseries #1}]}{\end{trivlist}}

\def\bL{\mathbf{L}}
%\usepackage{mathtime}

%%UNCOMMENT following line if you have package
\usepackage{times}

\title{ Nonparametric Covariance Estimation via Cholesky Decomposition with Shrinkage Toward Stationary Autoregressive Models}

\author{Tayler A. Blake\thanks{The Ohio State University, 1958 Neil Avenue, Columbus, OH 43201} \and  Yoonkyung Lee\thanks{The Ohio State University, 1958 Neil Avenue, Columbus, OH 43201}}
\begin{document}
\bibliographystyle{plainnat}
\maketitle

\begin{abstract}
With high dimensional longitudinal and functional data becoming much more common, there is a strong need for methods of estimating large covariance matrices. Estimation is made difficult  by the instability of sample covariance matrices in high dimensions and a positive-definite constraint we desire to impose on estimates. A Cholesky decomposition of the covariance matrix allows for parameter estimation via unconstrained optimization as well as a statistically meaningful interpretation of the parameter estimates. Regularization improves stability of covariance estimates in high dimensions, as well as in the case where functional data are sparse and individual curves are sampled at different and possibly unequally spaced time points. By viewing the entries of the covariance matrix as the evaluation of a continuous bivariate function at the pairs of observed time points, we treat covariance estimation as bivariate smoothing. 

Within regularization framework, we propose novel covariance penalties which are designed to yield natural null models presented in the literature for stationarity or short-term dependence. These penalties are expressed in terms of variation in continuous time lag and its orthogonal complement. We present numerical results and data analysis to illustrate the utility of the proposed method. \\
\\
%\begin{keywords}
{\bf keywords:} non-parametric, covariance, longitudinal data, functional data, splines, reproducing kernel Hilbert space
%\end{keywords}
\end{abstract}


\section{Introduction}

\indent

An estimate of the covariance matrix or its inverse is required for nearly all statistical procedures in classical multivariate data analysis, time series analysis, spatial statistics and, more recently, the growing field of statistical learning. Covariance estimates play a critical role in the performance of techniques for clustering and classification such as linear discriminant analysis (LDA), quadratic discriminant analysis (QDA), factor analysis, and principal components analysis (PCA), analysis of conditional independence through graphical models, classical multivariate regression, prediction, and Kriging. Covariance estimation with high dimensional data has has recently gained growing interest; it is generally recognized that there are two primary hurdles responsible for the difficulty in covariance estimation: the instability of sample covariance matrices in high dimensions and a positive-definite constraint we wish estimates to obey.

Prevalent technological advances in industry and many areas of science make high dimensional longitudinal and functional data a common occurrence, arising in numerous areas including medicine, public health, biology, and environmental science with specific applications including fMRI, spectroscopic imaging, gene microarrays among many others, presenting a need for effective covariance estimation in the challenging situation where parameter dimensionality $p$ is possibly much larger than the number of observations, $n$. 

We consider two types of potentially high dimensional data: the first is the case of functional data or times series data, where each observation corresponds to a curve sampled densely at a fine grid of time points; in this case, it is typical that the number of time points is larger than the number of observations. The second is the case of sparse longitudinal data where measurement times may be almost unique yet sparsely distributed within the observed time range for each individual in the study. In this case, the nature of the high dimensionality may not be a consequence of having more measurements per subject than the number of subjects themselves, but rather because when pooled across subjects, the total number of unique observed time points is greater than the number of individuals. 

Several approaches have been taken in effort to overcome the issue of high dimensionality in covariance estimation. Regularization improves stability of covariance estimates in high dimensions, particularly in the case where the parameter dimensionality $p$ is much larger than the number of observations $n$. Regularization of the covariance matrix and its Cholesky decomposition has been explored extensively through various approaches including banding, tapering, kernel smoothing, penalized likelihood, and penalized regression; see \citet{pourahmadi2011covariance} for a comprehensive overview. 
	
To overcome the hurdle of enforcing covariance estimates to be positive definite, several have considered modeling various matrix decompositions including variance-correlation decomposition, spectral decomposition, and Cholesky decomposition. The Cholesky decomposition has received particular attention, as it which allows for a statistically meaningful interpretation as well as an unconstrained parameterization of elements of the covariance matrix. This parameterization allows for estimation to be accomplished as simply as in least squares regression. If we assume that the data follow an autoregressive process with (possibly) heteroskedastic errors, then the two matrices comprising the Cholesky decomposition, the Cholesky factor (which diagonalizes the covariance matrix) and diagonal matrix itself, hold the autoregressive coefficients and the error variances, respectively. The autoregressive coefficients are often referred to in the literature as the \emph{generalized autoregressive parameters}, or \emph{GARPs}, and the error variances are often called the \emph{innovation variances}, or \emph{IVs}.

In longitudinal studies, the measurement schedule could consist of targeted time points or could consist of completely arbitrary (random) time points. If either the measurement schedule has targeted time points which are not necessarily equally spaced or if there is missing data, then we have what is considered incomplete and unbalanced data. If the measurement schedule has arbitrary or almost unique time points for every individual so that at a given time point there could be very few or even only a single measurement, we must consider how to handle what we consider as sparse longitudinal data. We view the response as a stochastic process with corresponding continuous covariance function and the generalized autoregressive parameters as the evaluation of a continuous bivariate function at the pairs of observed time points rather than specifying a finite set of observations to be multivariate normal and estimating the covariance matrix. This is advantageous because it is unlikely that we are only interested in the covariance between pairs of observed design points, so it is reasonable to approach covariance estimation in a way that allows us to obtain an estimate of the covariance between two measurements at any pair of time points within the time interval of interest. 


	%% monotonicity constraint 
%% I am unsure about how much detail to provide here about the mechanics of other methods and how our methods differ from (improve) previous work
Through the Cholesky decomposition, we formulate covariance estimation as a penalized regression problem and propose novel covariance penalties designed to yield natural null models presented in the literature. By transforming the axes of the design points, we express these penalties in terms of two directions: the lag component and the additive component and characterize the solution coefficient function in terms of a functional ANOVA decomposition. Some have side-stepped the issue of high dimensionality by prescribing simple parametric models for the elements of the Cholesky decomposition. \citet{chen2011efficient}, \citet{pourahmadi1999joint}, and \citet{pourahmadi2002dynamic} have elicited stationary parametric models for the generalized autoregressive coefficients, letting the GARPs depend only on the distance between two time points. To induce the structural simplicity of such stationary models with the flexibility of a nonparametric approach, we penalize all functional components but that corresponding to the lag component so that the set of null models is comprised of stationary models. \cite{huang2007estimation} follow the hueristic argument presented in \cite{pourahmadi1999joint} that the generalized autoregressive parameters are monotone decreasing in as lag increases and set off-diagonal elements of either the covariance matrix or the Cholesky factor corresponding to large lags to zero. Rather than shrinking element of the Cholesky factor to zero after a particular value of $l$, we choose to softly enforce monotonicity in $l$ by penalizing order restriction as in the work of \citet{tibshirani2011nearly}. 
	
The rest of the paper is organized as follows: Section 2 summarizes the general penalized estimation approach and introduces the transformed design coordinates and penalties for stationarity and non-monotonicity. Section 3 presents a detailed discussion of optimization and computational issues. Section 4 presents a simulation study and a real example to examine the performance of our methods as well as others. Section 5 concludes with discussion and future work.	

\section{Covariance estimation: a review}

Parametric models are frequently used for modeling covariance structure in the longitudinal data setting. Simple models which depend on a small number of parameters  are commonly found in the literature such as those corresponding to compound symmetry and autoregressive models of order $k$, where $k$ is small. However, model misspecification can lead to considerably biased estimates. Alternately, several have proposed applying nonparametric methods directly to elements of the sample covariance matrix or a function of the sample covariance matrix. Diggle and Verbyla (1998) introduced a nonparametric estimator obtained by kernel smoothing the sample variogram and squared residuals.  Yao, Mueller, and Wang applied a local linear smoother to the sample covariance matrix in the direction of the diagonal and a local quadratic smoother in the direction orthogonal to the diagonal to account for the presence of additional variation due to measurement error.  {\needsparaphrased[REVIEW 2009 WU AND POURAHMADI  METHOD: banding the sample covariance matrix. Under the assumption of short range dependency, they show that their estimator converges to the true covariance matrix for a broad class of nonlinear processes.]} The estimates yielded by these approaches, however, are not guaranteed to be positive definite. 

To satisfy the positive-definiteness constraint, methods have been developed and applied to certain reparameterizations of the covariance structure. Chiu, Leonard, and Tsui modeled the matrix logarithm of the covariance matrix. Early nonparametric work using the spectral decomposition of the covariance matrix included that of Rice and Silverman (1991) which discussed smoothing and smoothing parameter choice for eigenfunction estimation for regularly-spaced data. Staniswalis and Lee (1998) extended kernel-based smoothing of eigenfunctions to functional data observed on irregular grids. However, when the data are sparse in the sense that there are few repeated within-subject measurements and measurement times are quite different from subject-to-subject, approximation of the functional principal component scores defined by the Karhunen-Loeve expansion of the stochastic process by usual integration is unsatisfactory and requires numerical quadrature. Many have explored regression-based approaches using the Spectral decomposition, framing principal components analysis as a least-squares optimization problem. Among many others, Zou, Hastie and Tibshirani (2006), imposed penalties on regression coefficients to induce sparse loadings. {\needsparaphrased[REVIEW THE METHODS OF HUANG, KAUFMAN, YAO HERE]}


Leveraging regression techniques for covariance estimation has recently received much attention.  {[\needsparaphrased OMIT THIS; INCLUDE BRIEF SUMMARIES OF THEIR TECHNIQUES BELOW. including \citet{bickel2008regularized} and \citet{huang2006covariance}}] have proposed nonparametric estimators of a specific covariance matrix (or its inverse) rather than the parameters of a covariance function. 

{\needsparaphrased[DISCUSS THE CHOLESKY PARAMETERIZATION OF THE INVERSE COVARIANCE MATRIX. REFERENCE THE EARLY INFLUENTIAL WORKS]}

Recently, many have considered a modified Cholesky decomposition (MCD) of the inverse of the covariance matrix. This decomposition also ensures positive-definite covariance estimates, and, unlike the Spectral decomposition whose parameters follow an orthogonality constraint, the entries in the MCD of the covariance matrix are unconstrained and have an attractive statistical interpretation as particular regression coefficients and variances.  One drawback we might note, however, is that the interpretation of  the regression model induced by the MCD assumes a natural (time) ordering among the variables in $Y$, which we henceforth assume to have mean $0$, whereas other decompositions are permutation-invariant.

\citet{kaufman2008covariance} assume a stationary process, restricting covariance estimates to a specific class of functions. They as well as  Huang, Liu, and Liu \cite{huang2007estimation} follow the hueristic argument presented by \cite{pourahmadi1999joint} that $\phi_{t,t-l}$ is monotone decreasing in $l$ and set off-diagonal elements of either the covariance matrix or the Cholesky factor corresponding to large lags to zero. 

As in \citet{huang2007estimation}, \citet{kaufman2008covariance}, and \citet{yao2005functional}, we treat covariance estimation as a function estimation problem where the covariance matrix is viewed as the evaluation of a smooth function at particular design points. 



{\needsparaphrased[DISCUSS THE CHOLESKY PARAMETERIZATION OF THE INVERSE COVARIANCE MATRIX. REFERENCE THE Liao, Park, Hannig, and Kang (2015) paper, discuss their estimators and the competitors they investigated.]}



\citet{yao2005functional} do not utilize the Cholesky parameterization, and their estimates are not guaranteed to be positive definite.  We combine the advantages of bivariate smoothing as in \citet{yao2005functional} with the added utility of the Cholesky parameterization in \citet{huang2007estimation}; in doing so, we present a flexible and coherent approach to covariance estimation, while simultaneously we ensuring positive definiteness of estimates.Rather than shrinking element of the Cholesky factor to zero after a particular value of $l$, we choose to softly enforce monotonicity in $l$ by using a hinge penalty as in the work of \citet{tibshirani2011nearly}. 

\section{Covariance Estimation via Bivariate Smoothing}

Assume that we have measurements on individual $i$, denoted $\bfy_i = \left( y_{i1}, y_{i2}, \dots, y_{i,n_i} \right)^T$, at times $t_{i1} < t_{i2} < \dots< t_{i,n_i}$, with $i=1,2,\dots,N$. Observed time points may be individual-specific and not necessarily on a regular grid. To present a comprehensive overview our estimation procedure, we begin with the representation of the inverse covariance matrix, $\Sigma^{-1}$, in terms of its Cholesky decomposition (see \citet{pourahmadi2007cholesky} for a detailed discussion). Decomposing the precision matrix in such a way allows for both an unconstrained parameterization and statistically meaningful interpretation of covariance parameters. For any positive definite matrix $\Sigma$, there exists a unique unit lower triangular matrix $\matT$ with diagonal entries equal to $1$ which diagonalizes $\Sigma$:

\begin{equation}
\nonumber \matT \Sigma \matT^T = \matD
\end{equation}
\noindent

The entries of $\matT$ and $\matD$ are easily interpretable if we consider regressing $y_{ij}$ on its predecessors: 

\begin{equation}
{y}_{ij}  = \sum_{k=1}^{j-1} \phi_{ijk} y_{ik} + \sigma_{ij}\epsilon_{ij} \label{data_ARmodel}
\end{equation}
\noindent
for $j=2,\dots,n_i$; we define $y_{i1}=\epsilon_{i1}$. Standard regression theory gives us that if $\lbrace \phi_{ijk} \rbrace$ are the coefficients of the linear least squares predictor of $y_{ij}$ based on its predecessors, then the prediction residuals $\bfe_i =\left( e_{i1}, e_{i2},\dots, e_{i,n_i} \right)^T$ have diagonal covariance. Let $\matT_i$ be the unit lower triangular matrix with $jk^{th}$ below-diagonal entry given by $-\phi_{ijk}$. Let $\bfY_i$ denote the random vector giving rise to observed data $\bfy_i$, and let $\bfeps_i$ denote the associated vector of random errors. Then we may write the model \eqref{data_ARmodel} as follows: 

\begin{equation}
\bfeps_i = \matT_i \bfY_i \label{epsilon}
\end{equation}

We assume $\bfY_i$ centered to have mean $\bfo$ with covariance matrix $\Sigma_i$. Let $\matD_i$ be the diagonal matrix with $\lbrace \sigma_{ij} \rbrace$ down the diagonal, and taking covariances on both sides of \eqref{epsilon}, 

\begin{equation}
\nonumber
\matD_i = \matT_i \Sigma_i \matT_i^T
\end{equation} 
\noindent
and immediately, we have that $\Sigma_i^{-1} = \matT_i^T \matD_i^{-1} \matT_i$. The regression coefficients $\lbrace \phi_{ijk} \rbrace$ are referred to as the \emph{generalized autoregressive parameters} (GARPs), and the $\lbrace \sigma_{ij} \rbrace$ are referred to as the \emph{innovation variances} (IVs.) 

%% Show that the elements of the inverse covariance matrix can be viewed as conditional covariances 

Rather than a vector of longitudinal data points, we view the random vectors $\bfY_i$ and $\bfeps_i$ as discrete renditions of the stochastic processes: $Y\left(t\right)$ and $\epsilon\left(t\right)$.  We assume $Y\left(t\right)$ has corresponding covariance function $G\left(s,t\right)$ and that $\epsilon\left(s\right)$ follows a zero mean Gaussian white noise process with unit variance. It is reasonable to assume that if $\bfY$ is reasonably well-behaved, then $G\left(s,t\right)$ satisfies some smoothness conditions, where smoothness is defined in terms of square integrability of certain derivatives. We view the entries of $\Sigma_i$ as values of $G$ evaluated at the distinct pairs of observed time points on individual $i$. 

Additionally, we treat the elements of the precision matrix $\Sigma_i^{-1}$ as the values of the smooth function, $\gamma\left(s,t\right)$ evaluated at observed time points. If we consider the Cholesky decomposition of $\Sigma^{-1}$, it is natural to extent the same notion to the elements of $\matT_i$ and $\matD_i$: view the GARPs $\lbrace \phi_{ijk} \rbrace$ and innovation variances as the evaluation of the smooth functions $\phi\left(s,t\right)$ and $\sigma^2\left(t\right)$ at observed time points and interpret $\phi_{ijk} = \phi\left(t_{ij},t_{ik}\right)$ and $\sigma_{ij}^2 = \sigma^2\left(t_{ij}\right)$. 

Analogous to Pourahmadi's model \eqref{RV_ARmodel}, we model the continuous time process as follows: 
\begin{equation}   
y\left(t_{ij} \right)  = \sum_{k=1}^{j-1} \phi\left(t_{ij} ,t_{ik}\right) y\left(t_{ik}\right) + \sigma\left(t_{ij}\right)\epsilon\left({t_i}\right) \;\;\;\; i=1,\dots, N, 
\label{eq:MyModel} 
\end{equation}

It is advantageous to estimate the smooth function $\gamma\left(s,t\right)$ rather than a covariance matrix at a predetermined set of pairs of observed time points since observed time points may be unevenly spaced and vary from individual to individual. Several approaches to function estimation have been utilized in this setting; \cite{wu2003nonparametric}, for example, used locally weighted polynomials to smooth down the sub-diagonals of $\matT$. \cite{huang2007estimation} smoothed the sub-diagonals of $\matT$ using univariate smoothing splines.  Within our formulation, the task of estimating a covariance matrix is equivalent to estimating the function $\phi\left(s,t\right)$ using bivariate smoothing. For ease of exposition, we assume that $\sigma^2\left(t\right)$ is fixed and known. Like other nonparametric situations, we make no assumption about the functional form of $\phi$ other than that $\phi$ is smooth, with smoothness defined in terms of square integrability of certain derivatives and let $\phi$ belong to a reproducing kernel Hilbert space, $\mathcal{H}$.

%  Pooling the observed time points across subjects, we let $\mathcal{T}$ denote the set of all unique observed time points $\mathcal{T} = \bigcup \limits_{i=1}^{N} \bigcup\limits_{j=1}^{n_i}  \lbrace t_{ij} \rbrace$ and order them so that the elements of this set are given by $t_1 < t_2 < \dots < t_{p}$, $\vert \vert \mathcal{T} \vert \vert = p$. Let $\bfY = \left(Y_{t_1}, Y_{t_2}, \dots, Y_{t_p}\right)^T$ denote the vector of random variables corresponding to the process $Y$ at each of the unique 
%
%of pooled observations, and let $Cov\left(\bfY \right) = \Sigma$ where the $ij^{th}$ element of the $\Sigma$ is given by $\Sigma_{ij} = Cov\left(y_{t_i}, y_{t_j}\right)$. \\
%
% $\gamma$ is defined through $\phi$ and $\sigma$, which we also assume to be smooth functions.
%\begin{equation}
%{y}_{t_i}  = \sum_{j=1}^{i-1} \phi_{{t_i}{t_j}} y_{t_j} + \sigma_{t_j}\epsilon_{t_j} \label{RV_ARmodel}
%\end{equation}
%\noindent
%where $\bfphi_{t_i} = \left( \phi_{{t_i}{t_1}}, \phi_{{t_i}{t_2}}, \dots, \phi_{{t_i},{t_{i-1}}}\right)^T$ is the coefficient vector corresponding to the best linear predictor of $y_{t_i}$ based on its predecessors; Let $\matT$ be the $p \times p$ lower triangular matrix with unit diagonal and $\left(ij\right)^{th}$ element $-\phi_{ij}$, $i > j$ and $D$ be the diagonal matrix with diagonal entries $\sigma_1^2, \sigma_2^2, \dots, \sigma_p^2$. Using this notation, we may write 
%
%
%\noindent and letting $\text{L} = \matT^{-1}$, the modified Cholesky decomposition of $\Sigma^{-1}$ and $\Sigma$ are given by 
%\[
%\Sigma^{-1} = \matT^T \matD^{-1}\matT,\;\;\;\Sigma = \text{L} \matD \text{L}^T
%\]

%% begin discussion of estimation procedure, introduce data, replication
% Analogous to Pourahmadi's model \eqref{RV_ARmodel}, we model the continuous time process as follows: 
%\begin{equation}   
%{y}\left(t_i\right)  = \sum_{j=1}^{i-1} \phi\left(t_i ,t_j\right) y\left({t_j}\right) + \sigma\left(t_i\right)\epsilon\left({t_i}\right) \;\;\;\; i=1,\dots, n 
%\label{eq:MyModel} 
%\end{equation}

%The task of estimating a covariance matrix becomes the task of estimating the bivariate function $\phi\left(s,t\right)$ given noisy, discrete, and possibly unevenly spaced observations $Y_i= \left(Y\left(t_{i1}\right),Y\left(t_{i2}\right),\dots,Y\left(t_{i{n_i}}\right)  \right)^T$, $i=1,\dots,N$. The entries of the covariance matrix are viewed as the evaluation of this bivariate function at the unique observed pairs of time points. Like other nonparametric situations, we make no assumption about the functional form of $\phi$ other than that $\phi$ is smooth, with smoothness defined in terms of square integrability of certain derivatives.  
Along with \citet{huang2006covariance}, \citet{levina2008sparse}, and \citet{pourahmadi2000maximum} we consider the normal log-likelihood as a loss function, though it is important to note that the derivation of the Cholesky decomposition did not rely on any distributional assumption on $\bfeps$. Under the Gaussian assumption on $\epsilon\left(t\right)$, the negative log-likelihood of the data $\bfy_1,\bfy_2,\dots, \bfy_N$ up to a constant is given by

\begin{equation}
-2\lmr{L}\left(\bfy_1, \bfy_2, \dots,\bfy_N ,\Phi \right) = \sum_{i=1}^N \sum_{j=2}^{n_i} \sigma\left({t_j}\right)^{-2} \left(y\left({t_{ij}}\right) - \sum_{k=1}^{j-1}\phi\left({t_{ij},t_{ik}}\right)y\left({t_{ik}}\right) \right)^2 \label{loglikelihood}
\end{equation}


%%%%%%%%%%%%%%%%%%%%%%%%%%%%%%%%%%%%%%%%%%%%%%%%%%%%%%%%%%
%%%%%%%%%%%%%%%%%%%%%%%%%%%%%%%%%%%%%%%%%%%%%%%%%%%%%%%%%%
%%% Change this here to omit anything about second penalty
%%%%%%%%%%%%%%%%%%%%%%%%%%%%%%%%%%%%%%%%%%%%%%%%%%%%%%%%%%
%%%%%%%%%%%%%%%%%%%%%%%%%%%%%%%%%%%%%%%%%%%%%%%%%%%%%%%%%%


 We define our estimator $\hat{\phi}\left(s,t\right)$ to be the minimizer of the penalized log-likelihood. %We impose regularization in two stages: 


%\begin{itemize}
%\item {\bf Step 1:} Select $\hat{\lambda}_1$, where
\begin{equation} 
\hat{\phi} = \mathop{\mbox{arg min}}_{\phi} \left( -2\lmr{L} + \lambda_1 J_1\left(\phi\right) \right) \label{stage1obj}
\end{equation}
%\item {\bf Step 2:} Define $\hat{\phi}$ to the the minimizer of 
%\begin{equation}
%-2\lmr{L} + \hat{\lambda}_1 J_1\left(\phi^*\right) + \lambda_2 J_2\left(\phi\right) \label{stage2obj}
%\end{equation}
%\end{itemize}
\noindent


The first term in \eqref{stage1obj} discourages the lack of fit of $\phi$ to the data; $J_1$ is a penalty functional, and $\lambda_1$ is the smoothing parameter which controls the tradeoff between the lack of fit and amount of regularization imposed on $\hat{\phi}$ through $J_1$.  $J_1$ denotes the penalty assigned to the amount of ``non-stationarity'' in $\phi\left(s,t\right)$, or rather, any functional component that cannot be described in terms of the difference between the two argument values, $s-t$, $s \ge t$. %The second penalty term $J_2$ penalizes adjacent pairs of within-subject observed time differences that violate monotonicity in coefficient function magnitude, that is, $J_2$ softly enforces that the effect of past observations is monotonically decreasing as the difference between observations increases, as measured by the magnitude of certain functional components.

\subsection{Shrinkage toward Toeplitz precision  structures}

Many have specified parsimonious parametric models for $\phi_{ijk}$ to overcome the issue of dimensionality. A commonly utilized approach in previous work is to model $\phi_{ijk} = z_{ijk}^T \gamma$ where $z_{ijk}$ is a vector of powers of time differences and $\gamma$ is a vector of unknown ``dependence'' parameters to be estimated. \citet{chen2011efficient}, \citet{lin2009robust}, \citet{pan2003modelling},  and \citet{pourahmadi1999joint} let 
\begin{equation}
z_{ijk}^T = \left(1, t_{ij} - t_{ik},\left( t_{ij} - t_{ik} \right)^2, \dots, \left(t_{ij} - t_{ik}\right)^{q-1} \right) \label{covmodel}
\end{equation}

%% discuss how specifying a covariance structure in terms of lag only is equivalent to using a Toeplitz covariance model; define the class of models and form of the matrix

Modeling the covariance in such a way is reduces a potentially high dimensional problem to something much more computationally feasible; if we model the innovation variances $\sigma^2\left(t\right)$ similarly using a $d$-dimensional vector of covariates, the problem reduces to estimating $q+d$ unconstrained parameters, where much of the dimensionality reduction is a result of characterizing the GARPs in terms of lag only. Modeling $\phi^*$ in such a way is equivalent to specifying a Toeplitz structure for $\Sigma$. A $p \times p$ Toeplitz matrix $M$ is a matrix with elements $m_{ij}$ such that $m_{ij} = m_{\vert i-j \vert}$ i.e. a matrix of the form

\begin{equation}
M = \begin{bmatrix} m_0 & m_1 & m_2 & \dots & m_{p-1}\\ m_1 & m_0 & m_1 & \dots & m_{p-2}\\m_2 & m_1 & m_0 & \dots & m_{p-3}\\ \vdots & \vdots & \vdots & \ddots & \vdots\\  m_{p-1} & m_{p-2} & m_{p-3} & \dots & m_0 \end{bmatrix} \label{toeplitz}
\end{equation}


To shrink $\phi$ toward the continuous analogue of these models \eqref{covmodel}, it is useful to consider transforming the pairs of time points and estimating the re-parameterized coefficient function. The rotated pairs of time points become $l = s-t$, $m = \frac{1}{2}\left(s+t\right)$; re-expressing $\phi$ in terms of these new arguments, our goal is to estimate

\begin{equation}
\phi^*\left(l,m\right) = \phi^*\left(s-t, \frac{1}{2}\left(s+t\right)\right) = \phi\left(s,t\right)
\end{equation}

When $\phi^*$ corresponds to the simple models of the form \eqref{covmodel}, the bivariate function may be written in terms of only its first argument. Writing $\mathcal{H} = \mathcal{H}_l \otimes \mathcal{H}_m$ as a tensor product of two Hilbert spaces for each of $l$ and $m$, the ANOVA decomposition of $\phi^*\left(l,m\right) $ as presented in \citet{gu2002smoothing} is given by 

\begin{equation}
\phi^*\left(l,m\right) = \mu^* + \phi_1^*\left(l\right) + \phi_2^*\left(m\right) + \phi_{12}^*\left(l,m\right)   \label{ANOVA}
\end{equation}
\noindent 

We let $\mathcal{H}_l = \mathcal{H}_m = W_2\left(0,1\right)$ where $W_2$ denotes the second-order Sobolev space:
\[
W_2\left(0,1\right) = \lbrace f: \;\;f, f^\prime \mbox{absolutely continuous}, \int_0^1 \left(f^{\left( 2 \right)}\right)^2 dt < \infty \rbrace
\]  

%Let $k_j\left(x\right) = B_j/{j!}$ for, where $B_j\left(x\right)$ is the $j^{th}$ Bernoulli polynomial defined according to the recursive relationship:
%
%\[
%B_0\left(x\right) = 1,\;\;\;\;\;\; \frac{d}{dx} B_j\left(x\right) = jB_{j-1}\left(x\right)
%\]
%\noindent
%Noting that $M_\nu B_r = \delta_{\nu-r}$, $W_m$ can be written as a direct sum of the $m$ orthogonal subspaces: $\lbrace k_r \rbrace_{r=0}^{m-1}$ and $W_m^1$.   Here, $\lbrace k_r \rbrace$ is the subspace spanned by $k_r$ and $W_m^1$ is the space orthogonal to $W_m^0 \equiv \lbrace 1 \rbrace \oplus \lbrace k_1 \rbrace \oplus \dots \oplus \lbrace k_{m-1} \rbrace$ which satisfies 
%\[
%W_m^1 = \lbrace f: M_\nu f = 0,\;\; \nu = 0,1,\dots, m-1\rbrace
%\]
 \noindent
The penalty functional $J_1$ induces a decomposition of $\mathcal{H}$ as follows: $\mathcal{H}_l = \mathcal{H}_l^0 \oplus \mathcal{H}_l^1$ and $\mathcal{H}_m = \mathcal{H}_m^0 \oplus \mathcal{H}_m^1$ where let $\mathcal{H}_l^0 =  \lbrace 1 \rbrace \oplus \lbrace k_1 \rbrace$, $\mathcal{H}_m^0 =  \lbrace  1 \rbrace$, and where $\lbrace k_r \rbrace$ denotes the subspace spanned by $k_r$. $\mathcal{H}_l^1$ and $\mathcal{H}_m^1$ are the subspaces orthogonal to $\mathcal{H}_l^0$ and $\mathcal{H}_m^0$, respectively:
\begin{eqnarray*}
\mathcal{H}_l^1 = \lbrace \phi^*_1: \int_0^1 {\phi_1^*}^{\left( \nu \right)}\left(l\right) dl = 0,\;\; \nu = 0,1\rbrace\\
\mathcal{H}_m^1 = \lbrace \phi^*_2: \int_0^1 \phi_2^* \left(m\right) dm = 0 \rbrace\\
\end{eqnarray*}
\noindent
Using the properties of tensor product spaces, we may decompose $\mathcal{H} = \mathcal{H}^0\oplus \mathcal{H}^1$ where
\begin{eqnarray*}
\mathcal{H}^0 &=& \lbrace 1 \rbrace \oplus \lbrace k_1 \rbrace\\
\mathcal{H}^1 &=& \mathcal{H}_l^1 \oplus \mathcal{H}_m^1 \oplus  \left[ \lbrace k_1 \rbrace  \otimes  \mathcal{H}_m^1 \right]  \oplus  \left[\mathcal{H}_l^1 \otimes  \mathcal{H}_m^1\right]   
\end{eqnarray*}

We may write the penalty functional in terms of the projection of $\phi^* \in \mathcal{H}$ onto the penalized space of functions, $\mathcal{H}_1$:

\begin{eqnarray} 
\lambda_1 J_1\left(\phi\right) &=& \lambda_1 \left(\vert \vert {P_1 \phi_1^*} \vert \vert^2 + \vert \vert {P_1 \phi_2^*} \vert \vert^2 + \vert \vert {P_1 \phi_{12}^*} \vert \vert^2 \right)\\
&=& \lambda_1 \left(\vert \vert {\phi_1^*}^{\prime \prime} \vert \vert^2 + \vert \vert {\phi_2^*} \vert \vert^2 + \vert \vert {\phi_{12}^*} \vert \vert^2 \right) \label{nonstapen}
\end{eqnarray} 
 \noindent
$P_1 \phi^*$ denotes the projection of $\phi^* \in \mathcal{H}$ onto $\mathcal{H}_1$. To find the solution $\hat{\phi^*}$ which is the stage-wise minimizer of \eqref{objfun2}: we first set $\lambda_2 = 0$ and find $\tilde{\phi}^*$ which minimizes \eqref{objfun1}:

\begin{equation}
-2\lmr{L} + \lambda_1 J_1\left(\phi^*\right) = \sum_{i=1}^N \sum_{j=2}^{p_i} \sigma\left({t_j}\right)^{-2} \left(y\left({t_{ij}}\right) - \sum_{k=1}^{j-1}\phi\left({t_{ij},t_{ik}}\right)y\right)^2 + \lambda_1 \left(\vert \vert {P_1 \phi_1^*} \vert \vert^2 + \vert \vert {P_1 \phi_2^*} \vert \vert^2 + \vert \vert {P_1 \phi_{12}^*} \vert \vert^2 \right) \label{stage1obj}
\end{equation}
Let $\tilde{\lambda}_1$ denote the value of $\lambda_1$ corresponding to the minimizer $\tilde{\phi}^*$. Observe that large values of $\lambda_1$ (i.e. when $\lambda_1 \rightarrow \infty$) forces a parametric model in the null space of $J_1\left(\phi^*\right) = \lbrace \phi^*: \phi^*\left(l,m\right) = \alpha + \beta l \rbrace$, the set of $\phi^*$ to which $J_1$ assigns zero penalty. By construction, these nulls models correspond to modeling $\phi^*$ as linear functions of lag, so that as $\lambda_1 \rightarrow \infty$, the minimizer of \eqref{objfun1} corresponds to those models previously proposed \eqref{cov} for the case where $q=2$. 

Before we discuss the precise details of finding the minimizer of \eqref{stage1obj}, we introduce some notation: let $\mathcal{H}$ be endowed with inner product $\big < f,g\big >$, and define the reproducing kernel for $\mathcal{H}_1$  
\begin{eqnarray} \nonumber
R_l^1\left(l,l^\prime\right) &=& k_2\left(l \right)k_2\left(l^\prime \right) - k_{4}\left(\left[ l-l^\prime \right] \right)\\ \nonumber
R_m^1\left(m,m^\prime\right) &=& k_1\left(m \right)k_1\left(m^\prime \right) + k_2\left(m \right)k_2\left(m^\prime \right) - k_{4}\left(\left[ m-m^\prime \right] \right)\\
R^1\left(\left(l,m\right),\left(l^\prime,m^\prime\right)  \right) &=& \left[k_1\left(l\right)k_1\left(l^\prime \right) +  R_l^1\left(l,l^\prime\right)\right] R_m^1\left(m,m^\prime\right) \label{TPRK}
\end{eqnarray}
where $k_\nu = B_\nu/\nu!$ are scaled Bernoulli polynomials satisfying $B_0\left(x\right) = 1$, $\frac{d}{dx} B_j\left(x\right) = jB_{j-1}\left(x\right)$, and where $\left[ \alpha \right]$ is the fractional part of $\alpha$. Then we have the following result:

\begin{theorem}
Let $p= \sum_{i=1}^N {n_i \choose 2}$ be the total number of distinct within-subject pairs of design points, and index the transformed pairs $\left( l,m \right)_i$, $i=1, \dots,p$. Let $\matB$ be the $p \times 2$ matrix with $\left(i,j\right)^{th}$ entry $k_j\left( \left( l,m \right)_i \right)$ with rank $r=2$. Then, the unique minimizer of the penalized likelihood \eqref{penlik1}, $\phi^* \in \mathcal{H}$ is of the form

\begin{equation}
\phi^*\left(l,m\right) = d_0 + d_1 k_1\left(l\right) + \sum_{i=1}^{p} c_i R^1\left( \left(l,m\right) , \left(l,m \right)_i\right)
 \label{eq:finitedimsolution}
\end{equation}
\end{theorem}

{\bf Proof:}
Then we may verify that any $\phi^* \in \mathcal{H}$ can be written 
\[
\phi^*\left(l,m \right) = d_0 + d_1k_1\left(l\right) + \sum_{i=1}^n  c_i R_1\left( \left(l,m\right) , \left(l_i,m_i \right)\right) + \rho\left(l,m\right)
\]
\noindent
where $\rho \perp \mathcal{H}_0 = \lbrace 1\rbrace \oplus \lbrace k_1\rbrace,\; span\lbrace R_1\left(\left(l_i, m_i \right),\cdot \right)  \rbrace$. We do so by demonstrating that  $\rho$ does not improve the first term in \eqref{eq:objectivefun} (the data fit functional) and only adds to the penalty term, $J\left(\phi^*\right)$. Consequently, if $\hat{\phi^*}$ is the minimizer of \eqref{eq:objectivefun}, then $\rho = 0$. Using the properties of reproducing kernels, we can rewrite $\phi^*$ as an inner product of itself with $R$:
 
\begin{eqnarray*}
\phi^*\left(l_j,m_j \right)  &=& \left< R\left(\left(l_j,m_j\right),\left(\cdot,\cdot\right) \right),\phi^*\left(\cdot,\cdot\right)\right>\\
&=& \left<R_0\left( \left(l_j,m_j\right),\left(\cdot,\cdot\right) \right) + R_1\left(\left(l_j,m_j\right),\left(\cdot,\cdot\right) \right),d_0 + d_1k_1\left(\cdot \right)\right. \\ 
&\mbox{ }&\left. \;\;\;\;\;\;\;\;\;\;\;\;\;\;\;\;\;\;\;\;\;\;\;\;\;\;\;\;\;\;\;\;\;+ \sum_{i=1}^{N_{\phi^*}}  c_i R_1\left( \left(l_i,m_i \right),\left(\cdot,\cdot\right) \right) + \rho\left(\left(\cdot,\cdot \right)\right)\right>\\
&=& \left<R_0\left( \left(l_j,m_j\right),\left(\cdot,\cdot\right) \right) , d_0 + d_1k_1\left(\cdot\right)\right> + \left< R_0\left( \left(l_j,m_j\right),\left(\cdot,\cdot\right) \right),\sum_{i=1}^{N_{\phi^*}}  c_i R_1\left( \left(l_i,m_i \right),\left(\cdot,\cdot\right) \right)\right> \\
&\mbox{ }& + \left<R_0\left( \left(l_j,m_j\right),\left(\cdot,\cdot\right) \right), \rho\left(\left(\cdot,\cdot \right)\right)\right> + \left<R_1\left(\left(l_j,m_j\right),\left(\cdot,\cdot\right) \right), d_0 + d_1k_1\left(\cdot \right)\right> \\
&\mbox{ }& + \left<R_1\left(\left(l_j,m_j\right),\left(\cdot,\cdot\right) \right),\sum_{i=1}^{N_{\phi^*}}  c_i R_1\left( \left(l_i,m_i \right),\left(\cdot,\cdot\right) \right) \right> + \left<R_1\left(\left(l_j,m_j\right),\left(\cdot,\cdot\right) \right), \rho\left(\left(\cdot,\cdot \right)\right)\right>\\
&=& \left<R_0\left( \left(l_j,m_j\right),\left(\cdot,\cdot\right) \right) , d_0 + d_1k_1\left(\cdot\right)\right> + \left<R_1\left(\left(l_j,m_j\right),\left(\cdot,\cdot\right) \right),\sum_{i=1}^{N_{\phi^*}}  c_i R_1\left( \left(l_i,m_i \right),\left(\cdot,\cdot\right) \right) \right> \\
&\mbox{ }& + \underbrace{\left<R_0\left( \left(l_j,m_j\right),\left(\cdot,\cdot\right) \right)  , \rho\left(\cdot,\cdot\right) \right>}_{0} + \underbrace{\left<R_1\left( \left(l_j,m_j\right),\left(\cdot,\cdot\right) \right)  , \rho\left(\cdot,\cdot\right) \right>}_{0}\\
&=& d_0 + d_1k_1\left(\cdot \right) + \sum_{i=1}^{N_{\phi^*}}  c_i R_1\left( \left(l_i,m_i \right),\left(l_j,m_j\right) \right)
\end{eqnarray*}
\noindent


Rewriting the data fit functional, we have that  
 \begin{eqnarray*}
&\mbox{ }&\sum_{i=1}^N \sum_{j=1}^{n_i} \sigma_{ij}^{-2} \left(y\left(t_{ij}\right) - \sum_{k=1}^{j-1} \phi^*\left(t_{ij}, t_{ik}  \right) y\left(t_{ik}\right)  \right)^2  \\ 
&=& \sum_{i=1}^N \sum_{j=1}^{n_i} \sigma_{ij}^{-2} \left(y\left(t_{ij}\right) - \sum_{k=1}^{j-1} \left< R\left(\left(l^i_{jk},m^i_{jk}\right),\left(\cdot,\cdot\right) \right),\phi^*\left(\cdot,\cdot\right)\right> y\left(t_{ik}\right)  \right)^2  \\
 \end{eqnarray*}
\noindent
which is free of $\rho$. Consider the contribution of any nonzero $\rho$ to $J\left(\phi^*\right)$: 
  
 \begin{eqnarray*}
 J\left(\phi^*\right) &=& \vert \vert  P_1\phi^* \vert \vert^2\\
 &=& \left< \sum_{i=1}^{N_{\phi^*}}  c_i R_1\left( \left(l_i,m_i\right),\left(\cdot,\cdot\right) \right) + \rho\left(\cdot,\cdot \right), \sum_{j=1}^{N_{\phi^*}} c_j R_1\left( \left(l_j,m_j\right),\left(\cdot,\cdot\right) \right) + \rho\left(\cdot,\cdot\right)\right> \\
 &=& \vert \vert \sum_{i=1}^{N_{\phi^*}}  c_i R_1\left(\left(l_i,m_i\right),\left(\cdot,\cdot\right) \right) \vert \vert^2 + \vert \vert  \rho \vert \vert^2 
 \end{eqnarray*}
\noindent
Thus, including $\rho$ in $\phi^*$ only increases the penalty without improving (decreasing) the data fit functional, so we indeed have that the minimizer of \eqref{eq:objectivefun} has the form
\begin{equation}
 \phi^*\left(l,m\right) =  d_0 + d_1k_1\left(l\right) + \sum_{i=1}^{N_{\phi^*}} c_i R_1\left( \left(l,m\right) , \left(l_i,m_i \right)\right)
 \label{eq:finitedimsolution}
 \end{equation}

This result can be seen as a special case of the representer theorem given in \cite{kimeldorf1971some} and gives us that the minimizer $\phi^*$ lies within a finite dimensional space despite the optimization being carried out over an infinite dimensional space, $\mathcal{H}$. 


\subsection{Computation: Shrinkage toward Toeplitz precision structures}
For ease of exposition, we present the optimization problem in matrix notation. Let $\matK$ be the $p \times p$ matrix with $ij^{th}$ entry given by $R^1\left( \left(l,m\right)_i, \left(l,m\right)_j \right)$, $\matB$ defined as in the previous theorem, and let $\matD$ be the $\left( \sum_{i=1}^N n_i - N\right) \times \left( \sum_{i=1}^N n_i - N\right)$ diagonal matrix with diagonal entries $\sigma_{t_{12}}^2, \sigma_{t_{13}}^2, \dots,\sigma_{t_{1n_1}}^2, \dots, \sigma_{t_{{N},{n_N}}}^2$. Lastly, let $\bigY = \left(y_{t_{11}}, y_{t_{12}}, \dots, y_{t_{1,n_1}},y_{t_{21}},\dots,y_{\textsc{\relsize{-4}\textsl{{$N,n_N$}}}}\right)^T$ denote the $\sum_{i=1}^N n_i \times 1$ vector of pooled observations, and let $\bigY_{\textsc{\relsize{-2}{\textsl{(-1)}}}}$ and $\bigY_{\textsc{\relsize{-2}{\textsl{(-n)}}}}$ denote the $\left( \sum_{i=1}^N n_i - N\right)\times 1$ vectors of pooled observations with the first within-subject and last within-subject observations omitted, respectively, holding the responses and corresponding regression covariates in \eqref{MyModel}.   Rewrite \eqref{stage1obj} as

\begin{eqnarray}
\nonumber
-2\lmr{L} + \lambda_1J_1\left(\phi^*\right) &=& \left(\bigY_{\textsc{\relsize{-2}{\textsl{(-1)}}}}  - \matZ \Phi \right)^T \matD^{-1} \left(\bigY_{\textsc{\relsize{-2}{\textsl{(-1)}}}} - \matZ \Phi \right) + \lambda_1 \bfc^T \matK \bfc \\
&=& \left(\bigY_{\textsc{\relsize{-2}{\textsl{(-1)}}}} - \matZ\left(\matB\bfd + \matK\bfc\right) \right)^T \matD^{-1} \left(\bigY_{\textsc{\relsize{-2}{(-1)}}}  - \matZ\left(\matB\bfd + \matK \bfc \right) \right) + \lambda_1 \bfc^T \matK \bfc 
\end{eqnarray}
\noindent 
for appropriately specified $\left(\sum \limits_{i=1}^N n_i - N \right) \times p$ design matrix $\matZ=\matZ\left(\bigY_{\textsc{\relsize{-2}{\textsl{(-n)}}}} \right)$ where $\Phi$ denotes the vector with elements given by $\phi\left(s,t\right)$ evaluated at the unique pairs of observed time points. Taking derivatives, we have that
 
\begin{eqnarray}
\frac{\partial}{\partial \bfc}\left[ -2\lmr{L} + \lambda_1 J_1\left(\phi^*\right)\right] &=& - \left(\matZ \matK \right)^T \matD^{-1}\left[\bigY_{\textsc{\relsize{-2}{\textsl{(-1)}}}} - \matZ \left(\matB \bfd + \matK \bfc \right) \right] + \lambda_1 \matK \bfc= 0 \label{eq:normaleq1}\\
\frac{\partial}{\partial \bfd}\left[ -2\lmr{L} + \lambda_1 J_1\left(\phi^*\right) \right]&=& - \left(\matZ \matB \right)^T \matD^{-1}\left[\bigY_{\textsc{\relsize{-2}{\textsl{(-1)}}}} - \matZ \left(\matB \bfd + \matK \bfc \right) \right] = 0 \label{eq:normaleq2}
\end{eqnarray}
\noindent
We may write $\matB$ according to its Q-R decomposition: 
\[
\matB  = \left[ \matQ_1 \; \matQ_2\right]\left[ \begin{array}{c} \matR \\ 0 \end{array}\right] = \matQ_1 \matR  
\]

\noindent
where the columns of $\text{Q}=\left[ \matQ_1 \; \matQ_2\right]$ are orthogonal, with $Q_1$ having dimension $p \times 2$. Using this, we can show that the solutions to the normal equations \eqref{normaleq1}, \eqref{normaleq2} are given by 
\begin{eqnarray*}
\tilde{\bfc} &=&  \matQ_2\left[ \matQ_2^T\left[\matK + \lambda_1\left( \matZ^T \matD^{-1} \matZ \right)^{-1}\right] \matQ_2 \right]^{-1} \matQ_2^T\left(\matZ^T \matD^{-1} \matZ \right)^{-1}\matZ^T \matD^{-1} \bigY_{\textsc{\relsize{-2}{\textsl{(-1)}}}}\\
\tilde{\bfd} &=& \matR^{-1}\matQ_1^T\left[ \left( \matZ^T \matD^{-1} \matZ \right)^{-1}\matZ^T \matD^{-1} \bigY_{\textsc{\relsize{-2}{\textsl{(-1)}}}} -  \left[\matK + \lambda_1\left( \matZ^T \matD^{-1} \matZ \right)^{-1}\right] \hat{\bfc} \right]
\end{eqnarray*}
\noindent
Using some model selection criterion, such as the $K$-fold cross validation function, we select the optimal value of $\lambda_1$, denoted $\hat{\lambda}_1$. A more detailed discussion of smoothing parameter selection is presented in section four.
%
%We fix $\lambda_1 = \hat{\lambda}_1$ at the value of $\lambda_1$ for which $\left( \tilde{\bfc}\left(\lambda_1\right),\tilde{\bfd}\left(\lambda_1\right)\right) $ minimize $-2\lmr{L} + \lambda_1 J_1\left({\phi}^*\right)$. 
















\section{Shrinkage toward banded inverse structures}

%See \cite{rothman2010new} for proof that if $\matT$ is k-banded, then $\Sgima^{-1}$ is k-banded, and explain why this is attractive for regularization of elements on the inverse covariance matrix. Also explain using conditional covariances why it is attractive to have conditional dependence as defined through the magnitude of $\phi_1^*$ diminishing as $l$ increases.

The regularization instituted by $J_1$ determines the  complexity of those functional components which extent the stationary models, or rather the set of models that may be written in terms of an overall mean and the main effect of $l$ only. Several other ways of imposing structural simplicity have been explored in previous works; specifically, \cite{pourahmadi1999joint} was one of the first to present a hueristic argument that the GARPs, $\phi_{t,t-l}$ should be monotonically decreasing in $l$. That is, the effect of $y_{t-l}$ on $y_t$ through the autoregressive parameterization should decrease as the distance in time between the two measurements increases. They and others (see \cite{bickel2008regularized}, \cite{huang2007estimation},\cite{levina2008sparse}) enforce this structure by setting $\phi_{t,t-l}=0$ for $l > K$, or equivalently, setting all off-diagonals of $\matT$ beyond the $K^{th}$ off-diagonal to $0$, where $K$ is chosen using a model selection criterion such as AIC or BIC (see  \citet{wu2003nonparametric}.) This regularization is equivalent to banding the Cholesky factor $\matT$, or rather, regressing $y_t$ as in \eqref{data_ARmodel} on only its $K$ immediate predecessors, setting $\phi_{ijk} = 0$ for $j-k>K$. We may show that banding $\matT$ to only its first $K$ off-diagonals is equivalent to banding $\Sigma^{-1}$ to its first $K$ off-diagonals, and is a reasonable approach to imposing parsimony in the inverse covariance matrix, as off-diagonal zeros imply conditional independence between $y_t$ and $y_{t-l}$ given the intermediate observations under the assumption of Gaussian likelihood for any $l > K$. For this to become immediate, we need the following two propositions:

\begin{proposition}
Let $\bigY = \left(Y_1, \dots, Y_n\right)^T$ have joint distribution with mean vector $\bfmu$ and covariance matrix $\Sigma$. The elements of $\Sigma^{-1} = \left( \sigma^{ij} \right)_{1 \le i,j \le n}$ may be interpreted as partial covariances between the elements of $\bigY$.
\end{proposition}

\underline{\bf Proof:} We may easily derive the covariance between two measurements $Y_{j}$ and $Y_{k}$ conditional on the remaining measurements $\lbrace Y_{l} \vert l\ni j,k \rbrace$ using properties of the Normal distribution. Partition $\bigY = \left(Y_{1},Y_{2},\dots,Y_{,n}\right)^T$ as follows: letting $\bigY_{{\relsize{-2}{\textsl{-(jk)}}}}$ denote the full random vector with the $j^{th}$ and $k^{th}$ observation omitted, we let

\[
\bigY = \left[ \begin{array}{c}  Y_{j}\\ Y_{k}\\ \hdashline \bigY_{{\relsize{-2}{\textsl{-(jk)}}}} \end{array}\right]
\]
\noindent
which leads to the corresponding partition of $\Sigma = Cov\left(\bigY\right)$:

\begin{equation}
\Sigma = \left[\begin{array}{c:c}  \Sigma_{jk} &  \Sigma_{{\relsize{-2}{(jk),\textsl{-(jk)}}}} \\ \cdashline{1-2}  \hdashline \Sigma_{{\relsize{-2}{\textsl{-(jk)}},(jk)}}  & \Sigma_{{\relsize{-2}{\textsl{-(jk)}}}}   \end{array}\right] \equiv \left[\begin{array}{c:c} \Sigma_{11} & \Sigma_{12}  \\ \cdashline{1-2}  \hdashline \Sigma_{21} & \Sigma_{22}  \end{array}\right] \label{partition}
\end{equation}

Conditional distributions are easily obtained for the multivariate normal distribution; in particular, the covariance corresponding to the joint distribution $Y_{j}$ and $Y_{k}$ conditional on $\lbrace Y_{il} \vert l \ne j,k \rbrace$ is given by 

\begin{equation}
\Sigma_{11} - \Sigma_{12}^T\Sigma_{22}^T\Sigma_{12} = \Sigma_{jk} - \Sigma_{{\relsize{-2}{(jk),\textsl{-(jk)}}}}\Sigma_{22}^{-1}\Sigma_{{\relsize{-2}{\textsl{-(jk)}},(jk)}} \label{condcov}
\end{equation}
\noindent
Noting that the inverse of a matrix partitioned as in \eqref{partition} is given by 

\begin{equation}
\Sigma^{-1} \equiv \left[\begin{array}{c:c} \Sigma^{11} & \Sigma^{12}  \\ \cdashline{1-2}  \hdashline \Sigma^{21} & \Sigma^{22}  \end{array}\right]  = \left[\begin{array}{c:c}  \Sigma_{11} - \Sigma_{12}^T\Sigma_{22}^T\Sigma_{12}  & -\Sigma_{11}^{-1}\Sigma_{12}\left(\Sigma_{22} - \Sigma_{21}\Sigma_{11}^{-1}\Sigma_{12} \right) \\ \cdashline{1-2}  \hdashline -\Sigma_{22}^{-1}\Sigma_{21}\left(\Sigma_{11} - \Sigma_{12}\Sigma_{22}^{-1}\Sigma_{21} \right)  &  \Sigma_{22} - \Sigma_{21}^T\Sigma_{11}^T\Sigma_{21}  \end{array}\right] 
\end{equation}
\noindent
We may quickly see that the covariance of $Y_{j}$ and $Y_{k}$ conditional on $\lbrace Y_{il}\vert l\ne j,k \rbrace$ \eqref{condcov} corresponds to the upper left $2 \times 2$ sub-matrix of $\Sigma^{-1}$, $\Sigma^{11}$. 

\begin{proposition}
For any $p\times p$ positive definite matrix $\Sigma^{-1}$ and modified Cholesky decomposition $\Sigma^{-1} = \matT^T \matD^{-1}\matT$ where $\matT$ is unit lower triangular, for any column $j$ and $r\left(j\right)>j$, $\sigma^{pj} = \dots = \sigma^{r\left(j\right),j} = 0$ if and only if $t_{pj} = \dots = t_{r\left(j\right),j} = 0$ where $\lbrace \sigma^{ij}\rbrace$ and $\lbrace t_{ij}\rbrace$ denote the entries of $\Sigma^{-1}$ and $\matT$, respectively.
\end{proposition}

\underline{\bf Proof:} Using the expression
\[
\sigma^{ij} = \sum_{k=i}^p d_{ii}t_{ki}t_{kj}
\]
it follows immediately that $t_{pj} = \dots = t_{r\left(j\right),j} = 0$ implies that $\sigma^{pj} = \dots = \sigma^{r\left(j\right),j} = 0$.

From \cite{watkins2004fundamentals}, we can show that we can sequentially derive the elements of $\matT$ and $\matD$ according to 

\begin{eqnarray*}
d_{ii} = \sqrt{\sigma^{ii}-\sum_{k=1}^{i-1} t_{ki}^2 }\\
t_{ij} = \frac{1}{d_{ii}}\left(\sigma^{ij} - \sum_{k=1}^{i-1} t_{ki}t_{kj} \right)
\end{eqnarray*}
\noindent

We proceed by induction. For the first row of $\matT^T$, 





Regularization via banding the inverse covariance matrix is a generalization of tapering; \cite{cai2010optimal} define the $k$-tapered estimator of a $p \times p$ covariance matrix $\Sigma$ by $\tilde{\Sigma}^{\left(k\right)} = \left( \tilde{\sigma}_{ij}^{\left(k\right)} \right)_{1 \le i,j\le p} = \left( w_{ij}^{\left(k\right)}\hat{\sigma}_{ij} \right)_{1 \le i,j\le p}$ where $\hat{\Sigma}$ denotes the MLE for $\Sigma$ and where, for tapering parameter $k$,
\[
w_{ij}^{\left(k\right)} = \left\{ \begin{array}{ll} 1,& \vert i-j \vert \le k/2 \\ 2 - \frac{\vert i-j \vert}{k/2},& k/2 < \vert i-j \vert < k  \\  0, & otherwise    \end{array}\right.
\]

The $k$-banded estimator of $\Sigma$ may be defined in the same manner by specifying 
\[
\tilde{\sigma}_{ij}^{B\left(k\right)} = \hat{\sigma}_{ij}I\left( \vert i-j \vert \le k \right)
\]
Rather than tapering the covariance matrix itself, \cite{cai2012estimating} have generalized banding the inverse covariance matrix as in \cite{bickel2008regularized}, \cite{huang2007estimation}, and \cite{levina2008sparse} by proposing $k$-tapered estimators of $\Sigma^{-1}$.Rather than tapering elements of a covariance matrix given a specific set of observed time points, \cite{kaufman2008covariance} employ tapering to regularize a parametric estimator of a smooth covariance function, defining the tapered covariance function as follows:
\[
K_1\left(x, \bftheta, \gamma \right) = K_0\left(,\bftheta \right) K_{taper}\left( x,\gamma \right) 
\]

where $x$ is the distance between two observations, $K_0$ is the original covariance function which is assumed to be known up to a parameter vector $\bftheta \in \mathcal{R}^d$, and where $K_{taper}$ is an isotropic correlation function such that $K_{taper}\left(x,\gamma\right) = 0$ for $x \ge \gamma$. Rather than tapering the covariance matrix or a covariance function, \cite{cai2012estimating} have generalized banding the inverse covariance matrix as in \cite{bickel2008regularized}, \cite{huang2007estimation}, and \cite{levina2008sparse} by proposing $k$-tapered estimators of $\Sigma^{-1}$.

Thus, shrinking the Cholesky factor toward a unit diagonal structure is clearly a natural and desirable way to impose sparsity in the estimated inverse covariance matrix. We consider the form of the models belonging to the null space of $J_1$:

\begin{eqnarray*}
\mathcal{H}_0 &=& \lbrace \phi^* \vert \phi^* = \mu^* + \phi^*_1\left(l\right); \; \phi^*_l\left(l\right) = \beta_0 + \beta_1l \rbrace\\
&=& \lbrace \phi^* \vert \phi^* = d_0 + k_1\left(l\right) \rbrace\\
\end{eqnarray*}
\noindent

To further impose simplicity in the structure of the inverse covariance, we consider ``banding'' the functional components of these of these null models; specifically, if we consider any $\phi^*$ corresponding to a Toeplitz precision matrix, that is any $\phi^*$ of the form

\begin{equation}
\phi^*\left(l,m\right) = \mu^* + \phi^*_1\left(l\right)
\end{equation}
\noindent
we propose truncating the stationary functional components:  the overall mean and the main effect of $l$ to zero for any $l > l_0$ for some truncation point $l_0 \in \left(0,1\right)$. We consider the class of penalty functions that can be written in terms of an $L_p$ norm of the sum of the overall mean and the functional main effect of $l$. We follow in the work of \cite{huang2006covariance} and consider penalties which may be written

\begin{equation}
J_{2,\left(p\right)} = \sum_{\relsize{-2} l_i \in \mathcal{L}: l_i > l_0} \vert \mu^* + \phi^*_1\left(l_i\right) \vert^p  \label{bandedpenalty}
\end{equation}
\noindent

%\begin{equation}
%J_2 = \sum_{\relsize{-2} l_i \in \mathcal{L}: l_i > L} \vert \mu^* + \phi^*_1\left(l_i\right) \vert  \label{bandedpenalty}
%\end{equation}
%\noindent
where $\mathcal{L}$ denotes the observed values of $l$, so that any $\phi^*$ to which $J_2$ assigns zero penalty is one that inherits nonzero contribution from stationary functional components only for lags $l \le l_0$. We focus our attention to two important members of this family of penalties: the $L_2$ penalty and the $L_1$ penalty, given by
\begin{eqnarray} \nonumber
 J_{2,\left(2\right)} = \sum_{\relsize{-2} l_i \in \mathcal{L}: l_i > l_0} \left( \mu^* + \phi^*_1\left(l_i\right) \right)^2 \label{L2penalty} \;\;\mbox{and}\\
 J_{2,\left(1\right)} = \sum_{\relsize{-2} l_i \in \mathcal{L}: l_i > l_0} \vert \mu^* + \phi^*_1\left(l_i\right) \vert \label{L1penalty}
 \end{eqnarray}\noindent
respectively. These penalties will induce shrinkage in the autoregressive coefficient function $\phi^*$ (and hence in the overall inverse covariance function) as in ridge regression and LASSO, respectively. Considering both types of regularization introduced in this section and in the previous, any $\phi^*$ belonging to the set of models incurring zero penalty from both $J_1$ and $J_2$ may be written

\[
\phi^*\left(l,m\right) = \left\{ \begin{array}{lr} d_0 + d_1 k_1\left(l\right), & l \le l_0 \\ 0, & l > l_0 \end{array} \right.
\]

%The truncation penalty given by \eqref{bandedpenalty} sums over observed values of $l$ so as to approximate the integral 
%\[
%\int_{L}^1 \vert \mu^* + \phi_1^*\left(l\right) \vert dl
%\]


We impose this further regularization from a stage-wise approach and define $\hat{\phi}^*$ to be the minimizer of 

\begin{eqnarray}
&-2\lmr{L}& + \hat{\lambda}_1J_1\left(\phi^*\right) + \lambda_2 J_2\left(\phi^*\right)\\
= &-2\lmr{L}& + \hat{\lambda}_1J_1\left(\phi^*\right) + \sum_{\relsize{-2} l_i \in \mathcal{L}: l_i > l_0} \vert \mu^* + \phi^*_1\left(l_i\right) \vert  \label{stage2loss}
\end{eqnarray}
\noindent
%where $\hat{\lambda}_1$ is the optimal choice of tuning parameter value as determined by some model selection criterion; thorough discussion is reserved for Section four.

%Models chosen with tuning parameter selection determined using GCV have a desirable number of properties (see \cite{wahba1990spline} for detailed discussion.)

%Minimizing 
%\begin{eqnarray}
%-2\lmr{L} + \hat{\lambda_1}J_1\left(\phi^*\right) + \lambda_2 \sum_{\relsize{-2} l_i \in \mathcal{L}: l_i > L} \vert \mu^* + \phi^*_1\left(l_i\right) \vert  \label{stage2loss}
%\end{eqnarray}
%\noindent

We introduce two sets of non-negative variables $\big \lbrace \eta_{+,i} \big \rbrace$, $\big \lbrace \eta_{-,i} \big \rbrace$ so that minimizing \eqref{stage2loss} is equivalent to minimizing 
\begin{eqnarray}
-2\lmr{L} + \hat{\lambda}_1J_1\left(\phi^*\right) + \lambda_2 \sum_{\relsize{-2} l_i \in \mathcal{L}: l_i > l_0} \left( \eta_{+,i} + \eta_{-,i} \right)  \label{stage2loss}
\end{eqnarray}
\noindent subject to the constraints

\begin{eqnarray*}
\eta_{+,i},\; \eta_{-,i} &\ge& 0 \\
\mu^* + \phi^*\left(l_i\right) &\le& \eta_{i,+}\\
-\left(\mu^* + \phi^*\left(l_i\right)\right) &\le& \eta_{i,-}\label{eta_constraints}\\
\end{eqnarray*}
\noindent
for $i=1,\dots,\vert \vert\mathcal{L}\vert \vert-l_0$. Note that the solution $\hat{\phi}^*$, which depends on both $\lambda_1$ and $\lambda_2$, does not minimize the loss
\[
-2\lmr{L} + \lambda_1J_1\left(\phi^*\right) + \lambda_2J_2\left(\phi^*\right) 
\]
\noindent



Rather than simultaneously minimizing over the joint parameter space for $\lambda_1$ and $\lambda_2$: $\mathbb{R}^+\times\mathbb{R}^+$, we simplify the optimization problem by sequentially minimizing over the parameter space for $\lambda_1$ and,  given the optimal value for $\lambda_1= \hat{\lambda}_1$, the parameter space for $\lambda_2$. The choice of $l_0$ also determines the degree to which we truncate the inverse covariance function, or if we consider the precision matrix, which may be viewed as the discretized analogue to the smooth function, $l_0$ determines the degree to which we shrink the precision matrix toward a diagonal structure.

The effect of the tuning parameters $\lambda_1$ and $\lambda_2$ in addition to $l_0$ is most easily described if we consider the null models produced from this combination of penalty functions, or rather the form of the solution $\hat{\phi}^*$ as both $\lambda_1$ and $\lambda_2$ tend to infinity. First, if we let $\lambda_1 \rightarrow \infty$, then $\hat{\phi}^*$ may be written as a linear function of $l$. The precision matrix we may view as the discretized analogue of the corresponding smooth inverse covariance function in this case will be constant down each sub-diagonal, taking form \eqref{toeplitz}. Further, letting $\lambda_2 \rightarrow \infty$, $\phi^*$ is truncated to zero for any lag $l>L$, and elements of all subdiagonals of the corresponding precision matrix which lie further than $l_0$ from the diagonal are set to zero. Notice that if we let $l_0 = \mathop{\max}_{i} \lbrace l_i \rbrace$, the penalty $J_2$ is trivial, and no further structure is imposed on the solution $\hat{\phi}^*$ and the second stage of regularization requires no additional computational expense. 








\subsection{Computation: shrinkage toward banded Toeplitz precision structures via the $L_1$ penalty and constrained optimization }

In this section, we outline the derivation of the dual optimization problem for the optimal autoregressive coefficient function under the $L_1$ penalty \eqref{L1penalty}, though thorough explanation of the theory behind this derivation is omitted to allow for affable discussion.

Define the vector $\Phi^*_{1,J_2}$ with elements given by the functional main effect of $l$, $\phi^*_1$ evaluated at the penalized observed values of $l$, that is, $l_i\in \mathcal{L}$ such that $l_i > l_0$. Let $\matB_l$ and $\matK_l$ denote the $\left(\vert \vert \mathcal{L} \vert \vert-l_0\right) \times 2$ and $\left(\vert \vert \mathcal{L} \vert \vert-l_0\right) \times p$ matrices of basis functions and representers defined such that 
\begin{eqnarray*}
\mu^* \bfone + \Phi^*_{1,J_2} = \matB_l \bfd + \matK_l \bfc
\end{eqnarray*}

Using the standard machinery of primal-dual formulations in constrained optimization theory, four sets of Lagrange multipliers are introduced for the constraints \eqref{eta_constraints}: $\alpha_{+,i}$, $\alpha_{-,i}$, $\gamma_{+,i}$, and $\gamma_{+,i}$, $i=1,\dots, \vert \vert \mathcal{L}\vert \vert-l_0$ for $\mu^* + \phi^*\left(l_i\right) \le \eta_{i,+}$, $-\left(\mu^* + \phi^*\left(l_i\right)\right) \le \eta_{i,-}$, $\eta_{i,+} \ge 0$, and $\eta_{i,-} \ge 0$, respectively. The Lagrangian primal function is given by 

\begin{eqnarray}
\mathrm{l}_\mathcal{P} = \left(\bigY_{\textsc{\relsize{-2}{\textsl{(-1)}}}}  - \matZ \Phi \right)^T \matD^{-1} \left(\bigY_{\textsc{\relsize{-2}{\textsl{(-1)}}}} - \matZ \Phi \right) &+& \hat{\lambda}_1\bfc^T\matK\bfc + \lambda_2 \bfone^T\left( \bfeta_{+}+ \bfeta_{-}\right) - \bfalpha_+^T\bfeta_+ - \bfalpha_-^T\bfeta_  \\ \nonumber
&-& \bfgamma_+^T\left(\bfeta_+ - \left(\matB_l\bfd + \matK_l\bfc \right)  \right) - \bfgamma_-^T\left(\bfeta_- + \left(\matB_l\bfd + \matK_l\bfc \right)  \right)      \label{primal}
\end{eqnarray} \noindent
where 
\begin{eqnarray*}
\bfeta_+ &=& \left(\eta_{+,1}, \eta_{+,2},\dots,\eta_{\relsize{-5}+,\vert \vert \mathcal{L} \vert \vert-l_0 }\right)^T\\
\bfeta_- &=& \left(\eta_{-,1}, \eta_{-,2},\dots,\eta_{\relsize{-5}-,\vert \vert \mathcal{L} \vert \vert-l_0 }\right)^T\\
\bfalpha_+ &=& \left(\alpha_{+,1}, \alpha_{+,2},\dots,\alpha_{\relsize{-5}+,\vert \vert \mathcal{L} \vert \vert-l_0 }\right)^T\\
\bfalpha_- &=& \left(\alpha_{-,1}, \alpha_{-,2},\dots,\alpha_{\relsize{-5}-,\vert \vert \mathcal{L} \vert \vert-l_0 }\right)^T\\
\end{eqnarray*}

Define
\[
\bfa = \begin{bmatrix} \bfd\\ \bfc \end{bmatrix},\;\;\;\;\; \matH = \begin{bmatrix} \matB^T\matZ^T\matD^{-1}\matZ \matB & \bfo \\ \bfo & \hat{\lambda}_1\matK + \matK\matZ^T\matD^{-1}\matZ \matK   \end{bmatrix},\;\;\;\;\;\bfb = -2\bigY_{\textsc{\relsize{-2}{\textsl{(-1)}}}}^T\begin{bmatrix}\matZ\matB & \bfo \\ \bfo & \matZ\matK \end{bmatrix}
\]
\noindent
Then the Lagrangian primal function may be written
\begin{eqnarray}
\mathrm{l}_{\mathcal{P}} = -\bfb^T\bfa + \frac{1}{2}\bfa^T \matH \bfa + \lambda_2 \bfone^T\left( \bfeta_{+}+ \bfeta_{-}\right) &-& \bfalpha_+^T\bfeta_+ - \bfalpha_-^T\bfeta_{-} \label{lagrange_primal} \\ \nonumber
 &-& \bfgamma_+^T\left(\bfeta_+ - \left(\matB_l\bfd + \matK_l\bfc \right)  \right) - \bfgamma_-^T\left(\bfeta_- + \left(\matB_l\bfd + \matK_l\bfc \right)  \right) 
\end{eqnarray}

with the following constraints
\begin{eqnarray*}
\frac{\partial \mathrm{l}_{\mathcal{P}}}{\partial \bfa} &=& -\bfb + \matH\bfa + \left[ \matB_l\;\matK_L \right]^T \left(\bfgamma_+ - \bfgamma_- \right) = 0 \Leftrightarrow \matH^{-1}\left[ \bfb - \left[ \matB_l\;\matK_L \right]^T \left(\bfgamma_+ - \bfgamma_-\right) \right]  \\
\frac{\partial \eta_+}{\partial \bfa} &=& \lambda_2\bfone - \bfalpha_+ - \bfgamma_+ = 0 \Leftrightarrow \bfgamma_+ = \lambda_2\bfone - \bfalpha_+\\
\frac{\partial \eta_-}{\partial \bfa} &=& \lambda_2\bfone - \bfalpha_- - \bfgamma_- = 0 \Leftrightarrow \bfgamma_- = \lambda_2 \bfone - \bfalpha_-\\
\alpha_{+,i},\;&\alpha_{-,i}& \ge 0 \mbox{ and } \gamma_{+,i},\;\gamma_{-,i} \ge0 \mbox{ for }i=1,\dots,\vert \vert \mathcal{L}\vert \vert-l_0
\end{eqnarray*}

Simplifying the $\mathrm{l}_\mathcal{P}$ using the constraints, we have the dual problem of maximizing

\begin{eqnarray}
  -\frac{1}{2}\left[ \bfb - \left[\matB_l\;\matK_l\right]\left(\bfalpha_- -\bfalpha_+  \right) \right]^T\matH^{-1} \left[ \bfb - \left[\matB_l\;\matK_l\right]\left(\bfalpha_- -\bfalpha_+  \right) \right] \\
  \propto  \bfb^T\matH^{-1}\left[\matB_l\;\matK_l \right]^T\matD \bfalpha - \frac{1}{2}\bfalpha^T\matD^T \left[\matB_l\;\matK_l \right] \matH^{-1}\left[\matB_l\;\matK_l \right]^T\matD\bfalpha \label{dual}
\end{eqnarray} \noindent
with respect to $\bfalpha = \left[ \bfalpha_+^T\;\bfalpha_-^T \right]^T$ subject to
\begin{eqnarray}
\alpha_{+,i} &\le& \lambda_2\\
\alpha_{-,i} &\le& \lambda_2
\end{eqnarray}
\noindent
for all $i = 1,\dots, \vert \vert \mathcal{L}\vert \vert - l_0$. Note that the dual problem is a quadratic programming problem with non-negative definite matrix given by $Q \equiv \matD^T \left[\matB_l\;\matK_l \right] \matH^{-1}\left[\matB_l\;\matK_l \right]^T\matD$. We may also note that the optimization problem \eqref{dual} relies on $\vert \vert \mathcal{L}\vert \vert - l_0$ dual variables, so that the number of observed values of $l$ and choice of $l_0$ are responsible for determining the size of the problem rather than the parameter dimension. This is a great advantage computationally when the number of penalized observed values of $l$ is relatively small and the dimension of $\bfa$ is large. 

\subsection{Computation: shrinkage toward banded Toeplitz precision structures via the $L_2$ penalty}

\section{The truncated power basis and an alternative decomposition of $\mathcal{H}$}

The estimation of $\phi^*\left(l,m\right)$ is quite different from the usual problem of estimating an arbitrary bivariate function via smoothing. In the case of the latter, we most typically treat both arguments equally in terms of regularization, but in the case of covariance estimation and the generalized coefficient function equal treatment of $l$ and $m$ in terms of penalization perhaps is not the most appropriate approach. The lag component, $l$, has particularly significant meaning in terms of the covariance function and thus also in terms of $\phi^*$ and is of considerable more interest than the orthogonal component, $m$. As discussed in Section 2, we can define an entire class of stationary functional autoregressive models using only the $l$ direction, and additionally, as discussed in Section 3, there is a natural expectation about the functional form of the autoregressive coefficient function (and hence covariance) as a function of $l$, making imposing that conditional dependence between observation decay as $l$ and the time between observations increase a reasonable way to institute regularization.
%%motivate discretized penalties with the penalty functionals
This latter notion is instrumental in justifying the family of penalties
\[
J_{2,\left(p\right)} = \sum_{\relsize{-2} l_i \in \mathcal{L}: l_i > l_0} \vert \mu^* + \phi^*_1\left(l_i\right) \vert^p 
\]
\noindent
which we may view as a design-driven way of implementing the regularization which may be imposed by the penalty functionals taking the form

\begin{eqnarray} \nonumber
J\left(\phi^*\right) &=& \int_{l_0}^1 \vert \mu^* + \phi^*_1\left(l\right) \vert^p\; dl\\
&=& \int_{0}^1 \vert \mu^* + \phi^*_1\left(l\right) \vert^p I\left(l > l_0\right) \; dl \label{truncated_penalty}
\end{eqnarray}

Previously we decomposed the function space $\mathcal{H}$ according to $J_1 = \vert \vert {\phi^*}^{\prime\prime}_1 \vert \vert^2$ in a somewhat traditional sense, but the penalty functionals given by \eqref{truncated_penalty} motivate a different decomposition of $\mathcal{H}$. The form of \eqref{truncated_penalty} is significantly different in nature from the penalty discussed in Section 2.1 and those typically encountered in the setting smoothing spline ANOVA models, particularly because \eqref{truncated_penalty} effects only a subset of the domain for $l$. Therefore, an appropriate decomposition of the function space into the null space of $J$ and the penalized space should perhaps be formulated in terms of basis functions for the lag component, $l$ with domains which do not include the entire unit interval.

%%introduce truncated power basis typically presented in B-spline literature
\subsection{The truncated power basis}
%% introduce functional representation as a linear combination of truncated power basis functions
Consider a sequence of knots partitioning the unit interval $0 < x_1 < x_2 < \dots <x_n < 1$; the truncated power functions of degree $k$, $\lbrace T_{i,k} \rbrace_{i=1}^n$, are given by
\[
T^+_{i,k}\left(x\right) \left(x - x_i\right)^{k}_+ = \left\{\begin{array}{lr} \left(x-x_i\right)^{k-1}, & x-x_i \ge0\\ 0 & x-x_i < 0\end{array}\right.
\]

Polynomial regression splines are widely used in the nonparametric function estimation setting; these are functions are continuous piecewise polynomials where the definition of the function changes at the collection of knot points. A polynomial of degree $k$ has basis
\[
\lbrace 1,l,\dots,l^k, \left(l - l_1\right)^{k}_+,\dots, \left(l - l_n\right)^{k}_+ \rbrace
\]

%% Site de Boor paper here%%%%%%%%%%%%%%%%%%
A univariate function can be represented as a linear combination of these basis functions: 

\[
f = \sum_{j=0}^k \beta_j l^j + \sum_{i=1}^n \beta_{k+i} T_{i,k}
\]
\noindent
%\begin{figure}[h!]
%  \begin{center}
%      \includegraphics[scale=0.5,angle=270]{truncated_power_basis_functions.ps}
%  \caption{$T^+_{i,2}\left(l\right)=\left(\cdot - l_i \right)^2_+$ for $l_i = 0.2,0.4,0.5,0.6,0.8$}
%\end{center}
%\end{figure}

The truncated power basis, as in their use in defining polynomial regression splines, enjoy a particular ease of interpretation, as the coefficient $\beta_{i+k}$ may be identified as the size of the jump at $x_i$ in the $k^{th}$ derivative of $f$. This fact is especially useful when tracking change points or, in general, any abrupt changes in the regression curve. If we reflect these basis functions about each of their corresponding knot points and denote these reflections $\lbrace T^-_{ik}\rbrace$, then expressing the regularization corresponding to the penalty functionals \eqref{truncated_penalty} becomes quite natural in terms of the reflected basis functions $\left(\cdot - l_1 \right)^k_-,\dots, \left(\cdot - l_n \right)^k_-$, where $\left( \alpha \right)_- = \max\left(-\alpha,0\right)$.
%\begin{figure}[h!]
%  \begin{center}
%      \includegraphics[scale=0.5]{reflected_truncated_power_basis_functions.ps}
%  \caption{$T^-_{i2} =  \left(\cdot - l_i \right)^2_-$ for $l_i = 0.2,0.4,0.5,0.6,0.8$}
%\end{center}
%\end{figure}

Banding the inverse covariance structure becomes quite convenient by simply penalizing the regression coefficients corresponding to the reflected truncated power basis functions with support on any $l$ beyond some banding threshold, $l_0$. 
%\begin{figure}[h!]
%  \begin{center}
%      \includegraphics[scale=0.5,angle=270]{penalized_basis_functions.ps}
%  \caption{$\left(\cdot - l_i \right)^2_-$ for $l_i = 0.1, 0.15,0.2, \dots, 0.85,0.9$ with $l_0=0.5$}
%\end{center}
%\end{figure}

We consider a decomposition of $\mathcal{H}$ in terms of the reflected truncated power basis and corresponding penalty. We maintain the functional decomposition of $\phi^*\left(l,m\right) = \mu^* + \phi^*_1\left(l\right) + \phi^*_2\left(m\right) + \phi^*_{12}\left(l,m\right)$ we previously adopted, we preserve our definition of $\mathcal{H} = \mathcal{H}_l\otimes \mathcal{H}_m$ as a tensor product space where $\mathcal{H}_m = \mathcal{H}_m^0 \oplus \mathcal{H}_m^1$. For fixed $l_0\in\left(0,1\right)$, we decompose $\mathcal{H}_l = \mathcal{H}_l^0 \oplus \mathcal{H}_l^1$ where, for any $f \in \mathcal{H}_l$ we write

\begin{equation}
f\left(l\right) = \sum_{j=0}^k \beta_j l^j + \sum_{i=1}^n \beta_{k+i} T^*_{i,k}\left(l\right) \label{polynomial_expansion}
\end{equation} 

so that $\mathcal{H}_l = span\lbrace 1,l,\dots,l^k, T^*_{1k},\dots, T^*_{\vert \vert \mathcal{L} \vert \vert},k \rbrace$. We express $\mathcal{H}_m$ as the span of the same basis functions, and as before, we define $\mathcal{H}$ as the tensor product space $\mathcal{H} = \mathcal{H}_l \otimes \mathcal{H}_m$. The null models previously discussed lead to natural decompositions of $\mathcal{H}_l$ and $\mathcal{H}_m$: to shrink toward banded inverse covariance functions, we decompose $\mathcal{H}_l$ into two subspaces: $span\lbrace T^*_{ik}: l_i < l_0 \rbrace$ and $span\lbrace 1,l,\dots,l^k, T^*_{ik}: l_i \ge l_0 \rbrace$. For regularization resulting in null models corresponding to inverse Toeplitz structures, only a trivial decomposition of $\mathcal{H}_m$ is required: $span\lbrace 0 \rbrace$ and $span\lbrace 1,m,\dots,m^k,T^*_{1k},\dots,T^*_{\vert \vert \mathcal{M}\vert \vert,k} \rbrace$. The elements of 
%\begin{center}
%\begin{tabular}{c|cccccccc} 
%\relsize{-2}
%    & $T^*_{1k}$ & $T^*_{2k}$ & $\dots$ & $T^*_{\vert \vert \mathcal{M}\vert\vert,k}$ &  $\lbrace 1\rbrace$ & $\lbrace m \rbrace$ & $\dots$ & $\lbrace m^k\rbrace$ \\ \hline\\
%$T^*_{1k}$ & $T^*_{1k}\left(l\right)T^*_{1k}\left(m\right)$ &    &\dots & $T^*_{1k}\left(l\right)T^*_{\vert \vert \mathcal{M}\vert\vert,k}\left(m\right)$ & $T^*_{1k}\left(l\right)$ &    &\dots &$T^*_{1k}\left(l\right)m^k$ \\ 
%
%$T^*_{2k}$ & $T^*_{2k}\left(l\right)T^*_{1k}\left(m\right)$  &  &   & $\dots$ & $T^*_{2k}\left(l\right)$ & & & \\ 
%
%$\vdots$ & $\vdots$ & & & & $\vdots$ & & & \\ 
%
%$T^*_{i_0,k}$ & $T^*_{i_0,k}$ &     & $\dots$ &   &$T^*_{i_0,k}\left(l\right)$ & $\ddots$ &\dots & $T^*_{i_0,k}\left(l\right)m^k$ \\ \cdashline{1-1}
%$T^*_{i_0 + 1,k}$ & $T^*_{i_0 + 1,k}\left(l\right)$ &  & $\dots$ & $T^*_{i_0 + 1,k}\left(l\right)$ & $T^*_{i_0 + 1,k}\left(l\right)$&$T^*_{i_0 + 1,k}\left(l\right)$ &$\dots$ &$T^*_{i_0 + 1,k}\left(l\right)$ \\
%
%$\vdots$ & $\vdots$ & & $\dots$ & & $\vdots$ & & & \\
%
%$T^*_{\vert \vert \mathcal{L}\vert \vert,k}$ & & & $\dots$ & &$\dots$ & & & \\
%$\lbrace 1 \rbrace$ & $T^*_{2k}\left(m\right)$ &  & $\dots$ &  & $\dots$& & & $\lbrace m^k \rbrace$ \\
%
%$\lbrace l \rbrace$ & & & & & & & & \\
%$\vdots$ & & & & & & & & \\
%$\lbrace l^k \rbrace$ & & & & & & & & \\
%\end{tabular}
%\end{center}

\begin{center}
\begin{tabular}{c|c::ccc:ccc} 
			                  &   $\big \lbrace1\big \rbrace$ & $m$  & $\dots$ & $m^k$ & $T_{m_1,k}^-$ &    $\dots$ &  $T_{m_{\vert \mathcal{M}\vert},k}^-$   \\ \hline
&&&&&&&\\
$\big \lbrace 1 \big \rbrace$  		&  $\big \lbrace1\big \rbrace$ && ${\big \lbrace T_{m_{j},k}^- \big \rbrace}_{j=1}^{\vert \mathcal{M}\vert}$ &&& $\big \lbrace m^j \big \rbrace_{j=1}^k$ &\\ 
&&&&\\ \hdashline
 \hdashline
$\big \lbrace l \big \rbrace$  &&&&&&&\\
$\big \lbrace l^2 \big \rbrace$  &&&&&&&\\
$\vdots$  				& ${\big \lbrace l^i \big \rbrace}_{i=1}^{k}$&& $\big \lbrace l^im^j \big \rbrace_{i,j=1}^k$ & && $\big \lbrace l^iT_{m_{j},k}^- \big \rbrace_{i=1,j=1}^{k,\vert \mathcal{M}\vert}$&\\
$\big \lbrace l^{k-1} \big \rbrace$  &&&&&&&\\
$\big \lbrace l^k \big \rbrace$  &&&&&&&\\
\hdashline
$T_{l_1,k}^-$  			&  				 &&&&&&\\
$T_{l_2,k}^-$  			&   				 &&&&&&\\
$\vdots$  				& ${\big \lbrace T_{l_{i},k}^- \big \rbrace}_{i=1}^{i_0}$ && $\big \lbrace m^jT_{l_{i},k}^- \big \rbrace_{1=1,j=1}^{i_0,k}$ && & ${\big \lbrace T_{l_{i},k}^-T_{m_{j},k}^- \big \rbrace}_{i=1,j=1}^{i_0,\vert \mathcal{M}\vert}$ &\\
$T_{l_{i_0-1},k}^-$  		&   &&&&&&\\
$T_{l_{i_0},k}^-$  		& &&&&&&\\ \hdashline
$T_{l_{i_0+1},k}^-$  		& &&&&&&\\
$\vdots$  				&  ${\big \lbrace T_{l_{i},k}^- \big \rbrace}_{i=i_0+1}^{\vert \mathcal{L} \vert}$ && ${\big \lbrace m^jT_{l_{i},k}^- \big \rbrace}_{i=i_0+1}^{\vert \mathcal{L} \vert}$&&&${\big \lbrace T_{l_{i},k}^-T_{m_{j},k}^- \big \rbrace}_{i=i_0+1,j=1}^{\vert \mathcal{L} \vert,\vert \mathcal{M} \vert}$ &\\
$T_{l_{\vert \mathcal{L} \vert},k}^-$  &&&&&&&\\ 
\end{tabular}
\end{center}
\vspace{1in}

\begin{center}
\begin{tabular}{c|ccccccc:ccccccc} 
			                   &   & &  &$\big \lbrace  m^j  \big \rbrace_{j=0}^k$ & & & & & & &$\big \lbrace T_{m_j,k}^- \big \rbrace_{j=1}^{\vert \mathcal{M}\vert}$ & &  & \\ \hline
&&&&&&&&&&&&&& \\
&&&&&&&&&&&&&& \\
&&&&&&&&&&&&&& \\
$\big \lbrace  l^i  \big \rbrace_{i=0}^k$ &&&&$\big \lbrace \alpha_{ij} \big \rbrace$&&&&&&&$\big \lbrace \beta^\prime_{ij} \big \rbrace$&&&\\
&&&&$i=0,\dots,k$&&&&&&& $i=0,\dots,k$&&& \\
&&&&$i=0,\dots,k$&&&&&&&$j=1,\dots,\vert \mathcal{M}\vert$&&& \\
&&&&&&&&&&&&&&\\
&&&&&&&&&&&&&&\\ \hdashline
&&&&&&&&&&&&&& \\
$\big \lbrace T_{l_i,k}^- \big \rbrace_{i=1}^{i_0}$ &&&&&&&&&&&&&&\\
&&&&&&&&&&&&&& \\ 
&&&& $\big \lbrace \beta_{ij} \big \rbrace$ &&&&&&&$\big \lbrace \gamma_{ij} \big \rbrace$&&& \\
&&&&$i=1,\dots,\vert \mathcal{L} \vert$&&&&&&&$i=1,\dots,\vert \mathcal{L} \vert$&&& \\ 
&&&&$j=0,\dots,k$&&&&&&&$j=1,\dots,\vert \mathcal{M} \vert$&&& \\ 
$\big \lbrace T_{l_i,k}^- \big \rbrace_{i=i_0+1}^{\vert \mathcal{M}\vert}$ &&&&&&&&&&&&&&\\
&&&&&&&&&&&&&& \\ 
\end{tabular}
\end{center}



\vspace{5in}



\begin{center}
\begin{tabular}{c|ccccccc:ccccccc} 
			                   &   & &  &$\big \lbrace  m^j  \big \rbrace_{j=0}^k$ & & & & & & &$\big \lbrace T_{m_j,k}^- \big \rbrace_{j=1}^{\vert \mathcal{M}\vert}$ & &  & \\ \hline
&&&&&&&&&&&&&& \\
&&&&&&&&&&&&&& \\
&&&&&&&&&&&&&& \\
$\big \lbrace  l^i  \big \rbrace_{i=0}^k$ &&&&$\big \lbrace \alpha_{ij} \big \rbrace$&&&&&&&$\big \lbrace \beta^\prime_{ij} \big \rbrace$&&&\\
&&&&$i=0,\dots,k$&&&&&&& $i=0,\dots,k$&&& \\
&&&&$i=0,\dots,k$&&&&&&&$j=1,\dots,\vert \mathcal{M}\vert$&&& \\
&&&&&&&&&&&&&&\\
&&&&&&&&&&&&&&\\ \cdashline{2-8}
&&&&&&&&&&&&&& \\
$\big \lbrace T_{l_i,k}^- \big \rbrace_{i=1}^{i_0}$ &&&&&&&&&&&&&&\\
&&&&&&&&&&&&&& \\ 
&&&& $\big \lbrace \beta_{ij} \big \rbrace$ &&&&&&&$\big \lbrace \gamma_{ij} \big \rbrace$&&& \\
\hdashline
&&&&$i=1,\dots,\vert \mathcal{L} \vert$&&&&&&&$i=1,\dots,\vert \mathcal{L} \vert$&&& \\ 
&&&&$j=0,\dots,k$&&&&&&&$j=1,\dots,\vert \mathcal{M} \vert$&&& \\ 
$\big \lbrace T_{l_i,k}^- \big \rbrace_{i=i_0+1}^{\vert \mathcal{M}\vert}$ &&&&&&&&&&&&&&\\
&&&&&&&&&&&&&& \\ 
\end{tabular}
\end{center}

So that any $\phi^* \in \mathcal{H} = \mathcal{H}_l \otimes \mathcal{H}_m$ may be written

\begin{eqnarray*}
\phi^*\left( l,m \right) = \sum_{j=0}^k \beta_{lj}l^j + \sum_{j=0}^k \beta_{mj} m^{j} &+& \sum_{i=0}^k\sum_{j=0}^k\;\;\;\;\;\;\;\; l^i m^j  + \sum_{i=1}^{\vert \vert \mathcal{L}\vert \vert}\beta_{l,k+i} T_{ik}^*\left(l\right) + \sum_{i=1}^{\vert \vert \mathcal{M}\vert \vert}\beta_{m,k+i}T_{ik}^*\left(m\right) \\ 
 \sum_{j=0}^k\sum_{i=1}^{\vert \vert \mathcal{L}\vert \vert}\;\;\;\; \;\;\;\; m^j T_{ik}^*\left(l\right) &+& \sum_{j=0}^k\sum_{i=1}^{\vert \vert \mathcal{M}\vert \vert}\;\;\;\; \;\;\;\;   l^j T_{ik}^*\left(m\right) + \sum_{i=1}^{\vert \vert \mathcal{L}\vert \vert}\sum_{j=1}^{\vert \vert \mathcal{M}\vert \vert}\;\;\;\;\;\;\;\; T_{ik}^*\left(l\right)T_{jk}^*\left(m\right) \\
\end{eqnarray*}








%\begin{equation}
%J\left(\phi^*\right) = 
%\end{equation}
%%%% Instead of using 2, could formulate this in terms of general k as well







%\begin{eqnarray}
%R_l\left(l,l^\prime\right) &=& \sum_{j=0}^k l^k {l^{\prime}}^k + \sum_{i=1}^{\vert \vert \mathcal{L}\vert \vert} T^*_{i,k}\left(l\right)T^*_{i,k}\left(l^\prime\right)
%\end{eqnarray}
%
%The RKHS corresponding to $R_l$ is given by 
%
%\[
%\big \lbrace f: f = \sum_{j=0}^k \beta_jl^j + \sum_{i=1}^{\vert \vert \mathcal{L} \vert \vert} \eta_i T^*_{i,k}\left(l\right) \big \rbrace 
%\] 
%
%\noindent with inner product
%\begin{eqnarray}
%\big < f,g  \big >  &=& \big < \sum_{j=0}^k \beta_j l^j + \sum_{i=1}^{\vert \vert \mathcal{L} \vert \vert} \eta_i T^*_{i,k}\left(l\right), \sum_{j=0}^k \beta^\prime_j {l}^j +  \sum_{i=1}^{\vert \vert \mathcal{L} \vert \vert} \eta^\prime_i T^*_{i,k}\left(l\right)  \big > \\
%&=& \sum_{j=0}^k \beta_j\beta^\prime_j + \sum_{i=1}^{\vert \vert \mathcal{L} \vert \vert}  \eta_i \eta^\prime_i
%\end{eqnarray}
%\noindent with norm
%\begin{eqnarray}
%\vert \vert f \vert \vert^2 &=& \sum_{j=0}^k \beta_j^2 + \sum_{i=1}^{\vert \vert \mathcal{L} \vert \vert}  \eta_i^2\\
%&=& \vert \vert \bfbeta  \vert \vert^2 + \vert \vert \bfeta  \vert \vert^2
%\end{eqnarray}
%
%
%
%\begin{eqnarray}
%\mathcal{H}_l^0 &=&  \mathop{\bigoplus}_{l_i < l_0} \lbrace T_{i,3} \rbrace\\
%\mathcal{H}_l^1 &=& \lbrace 1 \rbrace \oplus \lbrace l \rbrace \oplus  \lbrace l^2 \rbrace \oplus \mathop{\bigoplus}_{l_i > l_0} \lbrace T_{i,3} \rbrace 
%\end{eqnarray}
%\noindent so that the elements of the tensor product between $\mathcal{H}_l$ and $\mathcal{H}_m$ are given by 
%
%\begin{tabular}{l|cc} 
%\\ \hline\\
%\end{tabular}

% Convergence results given in \cite{cai2010optimal} show that the tapering estimator is shown to be minimax rate optimal for estimating the bandable covariance matrices. Rather than tapering the covariance matrix itself, \cite{cai2012estimating} have generalized banding the inverse covariance matrix as in \cite{bickel2008regularized}, \cite{huang2007estimation}, and \cite{levina2008sparse} by proposing $k$-tapered estimators of $\Sigma^{-1}$. In addition to tapering estimators, others have utilized penalized likelihood methods to induce sparsity in precision matrix estimates. Particularly, the $l_1$-penalized likelihood and variants of the $l_1$-penalized likelihood have been considered in a number of papers, producing what shall be referred to as $l_1$ MLE-type estimators. \cite{huang2006covariance} and \cite{levina2008sparse} propose such estimators under Gaussian likelihood in terms of the Cholesky decomposition. \cite{yuan2007model} and \cite{friedman2008sparse}  %considering the class of models for the sparsity of the $p \times p$ precision matrix $\Omega$ given by 
%\[
%\mathcal{G}\left(c_{n,p},M_{n,p}\right) = \left \{  \begin{array}{c} \Omega = \left( \omega_{ij} \right)_{1 \le i,j\le p}: \mathop{\max}_{j} \sum_{i=1}^p \vert \omega_{ij} \vert^q \le c_{n,p};\\ 
%\vert \vert \Omega \vert \vert_1 \le M_{n,p},\; \lambda_{max}\left(\Omega\right)/\lambda_{min}\left(\Omega\right) \le M_1,\;\Omega \prec 0  \end{array}\right \}
%\]
%
%where $A \prec 0$ denotes that the matrix $A$ is symmetric and positive definite, $c_{n,p}$ and $M_{n,p}$ which depend on sample size $n$ and dimenionality $p$ are bounded away from $0$, and $M_1$ is a known constant. 

%We propose enforcing diminishing main effect of $l$ by defining $J_2$ as follows:
%
%\[
%\lambda_2 J_2\left(\phi^*\right) = \sum_{i=1}^{\vert \mathcal{L} \vert} \frac{\vert \phi^*_1\left(l_i\right) \vert}{1 - l_i}
%\]
%where $\mathcal{L} = \lbrace l_i \rbrace$ denotes the set of unique observed values of $l$. Note that the penalty associated with an observed lag $l_i$ tends to $\infty$ as $l_i \rightarrow 1$ when $\vert \phi^*_1\left(l_i\right) \vert > 0$.

%In addition to the structure imposed by $J_1$, we softly enforce monotonicity in $\vert \phi_1\left(l\right) \vert$, implying that the conditional dependence of $y\left(t\right)$ on $y\left(t-l\right)$ decreases as $l$ increases. To do this, we consider the penalty presented by \citet{tibshirani2011nearly}, which penalizes the fitted function at observed lags where $\phi_1^*$ is non-monotone decreasing in absolute value. We let 
%	
%	\begin{equation}
%	\lambda_2 J_2\left( \phi \right) =  \lambda_2 \sum_{i=1}^{\vert \vert \mathcal{L} \vert \vert - 1} \omega_i \frac{\left[ \vert\phi_1^*\left( l_{i+1} \right)\vert - \vert\phi^*_1\left( l_{i} \right)\vert\right]_+}{l_{i+1} - l_{i}}  \label{hingepenalty}
%	\end{equation}  
%	\noindent
%where $\mathcal{L}$ is the set of unique observed values of $l$, and 
%\[ \left(x\right)_+ = \left\{\begin{array}{lr}
%			   x & x \ge 0\\
%			   0 & x < 0
%\end{array}\right.
%\]
%\noindent
%The $\big \lbrace \omega_i \big \rbrace$ are a set of non-negative weights such that $\sum \limits_{i=1}^{\vert \vert \mathcal{L} \vert \vert} \omega_i = 1$. We define $\hat{\phi}$ to be the minimizer of 
%
%\begin{eqnarray}
%\nonumber
%-2\lmr{L} + \hat{\lambda}_1 J_1\left(\phi\right) + \lambda_2 J_2\left(\phi\right) = \sum_{i=1}^N \sum_{j=2}^{n_i}\sigma\left({t_j}\right)^{-2} \left(y\left({t_{ij}}\right) - \sum_{k=1}^{j-1}\phi\left({t_{ij},t_{ik}}\right)y\left({t_{ik}}\right) \right)^2\mbox{\;\;\;\;\;\;\;\;\;\;\;\;\;}\\  \mbox{\;\;\;\;\;\;\;\;\;\;\;\;\;\;\;} + \hat{\lambda}_1 J_1\left(\phi^*\right)  + \lambda_2 \sum_{i=1}^{\vert \vert \mathcal{L} \vert \vert - 1} \omega_i \frac{\left[\vert\phi_1^*\left( l_{i+1} \right)\vert - \vert\phi^*_1\left( l_{i} \right)\vert\right]_+}{l_{i+1} - l_{i}} \label{penlik1}
%\end{eqnarray}
%
%\noindent
%Note that an observed lag value, $l_i$,  contributes nothing to the hinge penalty only when $ \phi^*_1\left( l_{i+1}\right) - \phi_1^*\left( l_{i} \right) <  0$. To soften what one might consider overly rigid constraints of the previous inequality holding for all $i=1, \dots, \vert \vert \mathcal{L} \vert \vert-1$, we consider softly enforcing monotonicity of $\phi^*_1$ by introducing non-negative \emph{slack variables}, $\xi_i$, for each observed value of $l$ and reducing our constraints to
%
%\begin{equation} 
%\xi_i + \left(\phi^*_1\left( l_{i}\right) - \phi_1^*\left( l_{i+1} \right)\right) \ge  0,\;\;\;\;\;\;\;\xi_i > 0\;\;\;\;i = 1,\dots,\vert \vert \mathcal{L} \vert \vert-1 \label{slackconst}
%\end{equation}
% 
% If multiple $\xi_i$ are too large, the estimate of $\phi^*_1$ could exhibit much more non-monotonicity than desired. For example, if the $j$th observed value of $l$ is such that $\phi^*_1\left( l_{i}\right) < \phi_1^*\left( l_{i+1} \right)$, then for the $j$th constraint to be met, $\xi_j$ must be at least $\left(\vert\phi_1^*\left( l_{j+1} \right)\vert - \vert\phi_1^*\left( l_{j} \right)\vert\right)$. So, in some sense, $\sum_{i=1}^{\vert \vert \mathcal{L} \vert \vert - 1}\omega_i\xi_i$ provides an upper bound for the amount of non-monotonicity in $\phi^*_1$. The optimization of \eqref{2penloss} is then equivalent to minimizing 
%
%\begin{eqnarray}
%\nonumber
%-2\lmr{L} + \hat{\lambda}_1 J_1\left(\phi^*\right) + \lambda_2 J_2\left(\phi^*\right) =  \sum_{i=1}^{N} \sum_{j=1}^{n_i} \left(  y_{ij} - \sum_{k=1}^{j-1} \phi_{ik}y_{ik} \right)^2  
%+ \hat{\lambda}_1 J_1\left( \phi^* \right) +  \lambda_2 \sum_{i=1}^{\vert \vert \mathcal{L} \vert \vert -1} \frac{\omega_i}{l_{i+1}-l_i}\xi_i \label{xi_loss}\\
%\text{subject to }\xi_i > 0,\;\; \left(\vert\phi_1^*\left( l_{i+1} \right)\vert-\vert\phi^*_1\left( l_{i}\right)\vert\right) \le \xi_i  \mbox{\;\;\;\;\;\;\;\;\;\;\;\;\;\;\;\;\;\;\;\;\;\;\;\;\;\;\;\;} \label{xi_constraints}
%\end{eqnarray}
%\noindent
%for $i = 1, \dots, \vert \vert \mathcal{L} \vert \vert -1$. 
%
%
%
%\subsection{Constrained optimization: finding $\hat{\bfc}$, $\hat{\bfd}$}
%
% Let $\matDelta$ be the $\left(\vert \vert \mathcal{L} \vert \vert - 1 \right) \times \vert \vert \mathcal{L} \vert \vert $ differencing matrix, and let $\matK_{l}$ be the $p \times p$ matrix with $ij^{th}$ entry defined by the decomposition of $K=$ according to \eqref{TPRK}. Let $\matU$ be the matrix defined so that we may write the $\vert \vert \mathcal{L} \vert \vert -1$ constraints given by \eqref{xi_constraints} in the following matrix form: 
%\[
%\left[ \phi_1^*\left( l_{2}\right)-\phi^*_1\left( l_{1}\right),\phi_1^*\left( l_{3}\right)-\phi^*_1\left( l_{2}\right),\dots,\phi_1^*\left( l_{\textsc{\relsize{-2}{\textsl{$\vert\vert \mathcal{L}\vert\vert$}}}}\right)-\phi^*_1\left( l_{\textsc{\relsize{-2}{\textsl{$\vert\vert \mathcal{L}\vert\vert$ -1}}}}\right) \right]^T  =\matDelta \matU \begin{bmatrix}\matB& \matK_l \end{bmatrix} \begin{bmatrix} \bfd\\ \bfc \end{bmatrix} \le \bfxi
%\]
%\noindent
%where $\xi$ is the vector of $\vert \vert \mathcal{L} \vert \vert -1$ slack variables. Let $\bfomega = \left[ \frac{\omega_2}{l_2-l_1}, \frac{\omega_1}{l_3-l_2}, \dots,\frac{\omega_{\textsc{\relsize{-2}{\textsl{$\vert\vert \mathcal{L}\vert\vert$ -1}}}}}{l_{\textsc{\relsize{-2}{\textsl{$\vert\vert \mathcal{L}\vert\vert$}}}}-l_{\textsc{\relsize{-2}{\textsl{$\vert\vert \mathcal{L}\vert\vert$ -1}}}}} \right]^T$ and $\bfa = \begin{bmatrix} \bfd & \bfc \end{bmatrix}^T$. We introduce lagrangian terms $\bfalpha$, $\bfgamma$ for the constraints given in \eqref{xi_constraints} so that we may write \eqref{stage2obj} 
%
%\begin{equation}
%\lmr{l}_\mathcal{P} = -\bfb^T\bfa + \frac{1}{2} \bfa^T \matH \bfa + \bfomega^T \bfxi - \bfalpha^T  \left( \matC\bfa + \bfxi  \right) - \bfgamma^T \bfxi
%\end{equation}
%where $\matC = \matDelta \matU \begin{bmatrix}\matB& \matK_l \end{bmatrix}$ and $\bfa = \begin{bmatrix} \bfd^T & \bfc^T \end{bmatrix}^T$. Taking partial derivatives and setting them to zero, we have
%\begin{eqnarray}
%\frac{\partial}{\partial \bfa} \lmr{l}_{\mathcal{P}} &=& -\bfb + \matH\bfa - \matC^T\bfalpha \Leftrightarrow \bfa = \matH^{-1}\left(\matC^T\bfalpha + \bfb\right) \label{normeq1} \\
%\frac{\partial}{\partial \bfxi} \lmr{l}_{\mathcal{P}} &=& \bfomega \bfalpha - \bfgamma = 0 \Leftrightarrow \bfgamma = \bfomega - \bfalpha \label{normeq2}
%\end{eqnarray}
%
%Substituting \eqref{normeq1} and \eqref{normeq2} into $\lmr{l}_\mathcal{P}$, we can show that the dual problem as the form 
%\begin{equation}
%\mathop{\min}_{\bfalpha} -\frac{1}{2} \bfalpha^T\matC \matH^{-1}\matC^T\bfalpha + \bfb^T \matH^{-1}\matC^T\bfalpha\;\;\;\mbox{subject to }\bfo \le \bfalpha \le \bfomega 
%\end{equation}

%\[
%\bfb = -2\bigY_{\textsc{\relsize{-2}{\textsl{(-1)}}}}^T\begin{bmatrix} \matZ \matB & \bfo\\ \bfo & \matZ \matK \end{bmatrix}\;\;\;\;\;\matH = \begin{bmatrix} \matB^T \matZ^T \matD^{-1}\matZ \matB  & \bfo\\ \bfo & \hat{\lambda}_1\matK \matZ^T \matD^{-1}\matZ \matK \end{bmatrix}
%\]


%\begin{table}
%\caption{Genotypes and Their Genotypic Values for a Diallelic Locus Genotypes and Their Genotypic Values for a Diallelic Locus Genotypes and Their Genotypic Values for a Diallelic Locus Genotypes and Their Genotypic Values for a Diallelic Locus Genotypes and Their Genotypic Values for a Diallelic Locus }
%\begin{center}
%\begin{tabular}{ccccc}
%\hline
%\hline
%\\[-5pt]
%\multicolumn{2}{c}{Genotype} & &
%\multicolumn{1}{c}{Dummy for additivity} &
%\multicolumn{1}{c}{Dummy for dominance }\\
%\multicolumn{1}{c}{Label} &    
%\multicolumn{1}{c}{Index i} &
%\multicolumn{1}{c}{Genotypic value ($\eta$)}&
%\multicolumn{1}{c}{effect $\alpha$ (x)} &
%\multicolumn{1}{c}{effect $\delta$ (z)}\\
%\hline
%qq      &1&     $\mu + \mbox{2}\alpha$  & 2&    0\\
%Qq&     2&      $\mu + \alpha + \delta$&        1       &1\\
%QQ&     3&      $\mu$&  0&      0\\
%\hline
%\end{tabular}
%\end{center}
%\end{table}


\bibliography{Master}

\end{document}


